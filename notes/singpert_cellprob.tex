\documentclass[10pt]{article}
\oddsidemargin = 0.2in
\topmargin = -0.5in
\textwidth 6in
\textheight 8.5in

\usepackage{graphicx,bm,hyperref,amssymb,amsmath,amsthm}

% -------------------------------------- macros --------------------------
% general ...
\newcommand{\bi}{\begin{itemize}}
\newcommand{\ei}{\end{itemize}}
\newcommand{\ben}{\begin{enumerate}}
\newcommand{\een}{\end{enumerate}}
\newcommand{\be}{\begin{equation}}
\newcommand{\ee}{\end{equation}}
\newcommand{\bea}{\begin{eqnarray}} 
\newcommand{\eea}{\end{eqnarray}}
\newcommand{\ba}{\begin{align}} 
\newcommand{\ea}{\end{align}}
\newcommand{\bse}{\begin{subequations}} 
\newcommand{\ese}{\end{subequations}}
\newcommand{\bc}{\begin{center}}
\newcommand{\ec}{\end{center}}
\newcommand{\bfi}{\begin{figure}}
\newcommand{\efi}{\end{figure}}
\newcommand{\ca}[2]{\caption{#1 \label{#2}}}
\newcommand{\ig}[2]{\includegraphics[#1]{#2}}
\newcommand{\bmp}[1]{\begin{minipage}{#1}}
\newcommand{\emp}{\end{minipage}}
\newcommand{\pig}[2]{\bmp{#1}\includegraphics[width=#1]{#2}\emp} % mp-fig, nogap
\newcommand{\bp}{\begin{proof}}
\newcommand{\ep}{\end{proof}}
\newcommand{\ie}{{\it i.e.\ }}
\newcommand{\eg}{{\it e.g.\ }}
\newcommand{\etal}{{\it et al.\ }}
\newcommand{\pd}[2]{\frac{\partial #1}{\partial #2}}
\newcommand{\pdc}[3]{\left. \frac{\partial #1}{\partial #2}\right|_{#3}}
\newcommand{\infint}{\int_{-\infty}^{\infty} \!\!}      % infinite integral
\newcommand{\tbox}[1]{{\mbox{\tiny #1}}}
\newcommand{\mbf}[1]{{\mathbf #1}}
\newcommand{\half}{\mbox{\small $\frac{1}{2}$}}
\newcommand{\C}{\mathbb{C}}
\newcommand{\N}{\mathbb{N}}
\newcommand{\R}{\mathbb{R}}
\newcommand{\Z}{\mathbb{Z}}
\newcommand{\RR}{\mathbb{R}^2}
\newcommand{\ve}[4]{\left[\begin{array}{r}#1\\#2\\#3\\#4\end{array}\right]}  % 4-col-vec
\newcommand{\vt}[2]{\left[\begin{array}{r}#1\\#2\end{array}\right]} % 2-col-vec
\newcommand{\bigO}{{\mathcal O}}
\newcommand{\qqquad}{\qquad\qquad}
\newcommand{\qqqquad}{\qqquad\qqquad}
\DeclareMathOperator{\Span}{Span}
\DeclareMathOperator{\im}{Im}
\DeclareMathOperator{\re}{Re}
\DeclareMathOperator{\vol}{vol}
\newtheorem{thm}{Theorem}
\newtheorem{cnj}[thm]{Conjecture}
\newtheorem{lem}[thm]{Lemma}
\newtheorem{cor}[thm]{Corollary}
\newtheorem{pro}[thm]{Proposition}
\newtheorem{rmk}[thm]{Remark}
% this work...
\newcommand{\pO}{{\partial\Omega}}
\newcommand{\LpO}{\Delta_\pO}
\newcommand{\eps}{\epsilon}
\newcommand{\dn}{\partial_n}



\begin{document}

\title{Asymptotic expansion of a coupled surface-bulk diffusion problem in
  the reciprocal of the bulk diffusivity}

\author{Alex H. Barnett}
\date{\today}
\maketitle

\begin{abstract}
  In the spherical cell polarization model of Diegmiller et al (2018),
  the bulk diffusivity $D_C$ is taken as infinite.
  In fact $D_C$ is large but finite, being $1/\eps \sim 10^2$ times
  larger than the surface diffusivity.
  A full numerical solution of the coupled bulk and surface diffusion equations
  in a general cell geometry---either by finite difference or time-domain
  boundary integral methods---%
  would be daunting due to issues of discretization and possible small time steps.
  Here we apply the classical method of {\em singular perturbation} in powers of $\eps \ll 1$.
  The biologically relevant outer layer solution to a given order
  involves only bulk Poisson solves, so might make a more attractive numerical scheme.
\end{abstract}

\section{Introduction}

Let $\Omega \subset \R^3$ be a smooth bounded domain
representing a cell, with boundary $\pO$.
In \cite{diegmiller18} a cell polarization model is presented
with bulk concentration $C(x,t)$ diffusing in $\Omega$ with
diffusivity $D_C$,
coupled to a surface concentration $B(x,t)$ obeying a nonlinear
reaction-diffusion equation on $\pO$ with much lower diffusivity.
They consider the special case $\Omega$ a ball; we consider a general shape.
Non-dimensionalizing their model with length units of order the diameter,
and time units such that the surface diffusion is unity,
one gets the IBVP
\bea
\dot{B} - \LpO B   &=&  f(B,C)  \qquad \mbox{ on } \pO \times (0,\infty)
\label{surfpde}
\\
\eps \dot{C} - \Delta C   &=& 0   \qqquad \mbox{ in } \Omega \times (0,\infty)
\label{pde}
\\
\dn C  &=& -\eps f(B,C) \qquad \mbox{ on } \pO \times (0,\infty)
\label{fick}
\\
B(x,0) &=& B_i(x)  \qqquad x\in\pO
\label{Bi}
\\
C(x,0) &=& C_i(x)  \qqquad x\in\Omega ~,
\label{Ci}
\eea
where a dot indicates $\partial_t$, and $\dn = \mbf{n}\cdot\nabla$
where $\mbf{n}$ is the unit surface normal facing out of $\Omega$.
The given initial data are $B_i$ and $C_i$.
Here $\eps \ll 1$ is interpreted as $D_B/D_C$ (the ratio of surface
to bulk diffusivities), and in \cite{diegmiller18} is quoted to be about
$0.003$.
Fick's law relating Neumann boundary data to net flux is \eqref{fick}.
This flux is controlled by $f(B,C)$, a fixed function of two variables,
giving the local net flux from bulk to surface at a point on $\pO$, in terms of the local surface concentration $B$ and bulk concentration $C$ at that point.
It is generally nonlinear, the example from \cite{diegmiller18} being
$$
f(B,C) = (\beta + \frac{B^\nu}{B^\nu + \Gamma^\nu})C - k_d B~,
$$
where the two terms are binding and unbinding rates,
for constants $\beta$, $\Gamma$, and $k_d$. We will leave $f$ arbitrary.

The total amount of chemical (bulk plus surface) is conserved as follows.
%as follows \cite{diegmiller18}.
\begin{pro}[Conservation law]
  For any $\eps>0$,
  any solution of the above IBVP obeys
  \be
\int_\pO B(\cdot,t) + \int_\Omega C(\cdot,t) \; = \; A ~,  \qquad t>0~,
\label{cons}
\ee
with constant $A:=\int_\pO B_i + \int_\Omega C_i$.
\label{p:cons}
\end{pro}
\begin{proof}
Via the bulk and surface divergence theorems and the PDEs,
$$
\eps \int_\Omega \dot{C} = \int_\Omega \Delta C = \int_\pO \dn C =  -\eps
\int_\pO f(B,C) = -\eps \int_\pO (\dot{B}-\LpO B) = -\eps \int_\pO \dot{B}~,
$$
then one cancels $\eps$.
\end{proof}


% SSSSSSSSSSSSSSSSSSSSSSSSSSSSSSSSSSSSSSSSSSSSSSSSSSSSSSSSSSSSSSSSSSSSSSSSSSS
\section{Singular perturbation and expansions in $\eps$}

For small $\eps$ the above IBVP is a singularly-perturbed problem in time,
so it is standard to split its analysis into initial layer
and outer layer solutions
\cite[Sec.~3.4]{logan} \cite[Sec.~10.2]{linsegel}
(note we reverse their convention of capital letters for inner and small
letters for outer).
We will see that the former is a relaxation phase of $C$ where $B$ is be essentially constant,
while in the latter $B$ will also change and $C$ stays nearly constant as a function of space.
Note that only the outer layer solution, to zeroth order in $\eps$, is
considered in \cite{diegmiller18}.


% iiiiiiiiiiiiiiiiiiiiiiiiiiiiiiiiiiiiiiiiiiiiiiiiiiiiiiiiiiiiiiiiiiiiiiiiiii
\subsection{Initial layer solution at zeroth and first order}

Rewriting using the rescaled time $\bar{t}:=t/\eps$, the IBVP
becomes, using small letters for the inner solution,
\bea
\dot{b} - \eps\LpO b   &=&  \eps f(b,c)  \qquad \mbox{ on } \pO \times (0,\infty)
\label{surfpdel}
\\
\dot{c} - \Delta c   &=& 0   \qqquad \mbox{ in } \Omega \times (0,\infty)
\label{pdel}
\\
\dn c  &=& -\eps f(b,c) \qquad \mbox{ on } \pO \times (0,\infty)
\eea
with IC \eqref{Bi}--\eqref{Ci} as before.

One can then expand the solution as $b = b_0 + \eps b_1 + \eps^2 b_2 + \dots$,
etc, and match powers of $\eps$.

At zeroth order \eqref{surfpdel} becomes $\dot{b_0}=0$.
And \eqref{pdel} with homogeneous Neumann BCs $\dn c_0 = 0$ means that
the zeroth order inner layer solution is analytic:
\be
b_0(x,\bar{t}) = B_i(x)~,
\quad x\in\pO, \; \bar{t}>0~,
\qqquad
c_0(x,\bar{t}) = \bar{C}_i + \sum_{j=1}^\infty \alpha_j e^{-\lambda_j \bar{t}} \psi_j(x) ~,
 \quad x\in\Omega, \; \bar{t}>0
\label{BCl0}
 \ee
where $\bar{C}_i := (\vol\Omega)^{-1} \int_\Omega C_i$ is the average of the
initial data,
and $(\lambda_j, \psi_j)$ are the non-trivial Neumann Laplace eigenpairs
of $\Omega$, ordered such that $0<\lambda_1\le\lambda_2\le\dots$.
The coefficients are given by the Euler--Fourier formula
$\alpha_j = \langle \psi_j,C_i\rangle_{L^2(\Omega)}$, assuming
$\|\psi_j\|=1$ $\forall j$.
Note that at this order, $b_0$ is constant, whereas $c_0$ equilibrates on
a slowest $\bar{t}$ timescale of $1/\lambda_1$, which (unless $\Omega$
has peculiar features such as narrow necks) is $\bigO(1)$.
This relaxation time for $C$ will thus be $\bigO(\eps)$
back in the original variable $t$.

At first order, we get a linear IBVP (recalling time-derivatives
are with respect to inner time $\bar{t}$),
\bea
\dot{b_1}  &=& \LpO c_0 + f(b_0,c_0)  \qquad \mbox{ on } \pO \times (0,\infty)
\\
\dot{c_1} - \Delta c_1   &=& 0   \qqqquad \mbox{ in } \Omega \times (0,\infty)
\\
\dn c_1  &=& -f(b_0,c_0) \qquad \mbox{ on } \pO \times (0,\infty)
\\
b_1(x,0) &=& 0  \qqquad x\in\pO
\label{Bi1}
\\
c_1(x,0) &=& 0  \qqquad x\in\Omega ~,
\label{Ci1}
\eea
where $b_0$ and $c_0$ are given by \eqref{BCl0}.
The solution is pointwise linear growth of $b_1$, and, independently,
the linear heat equation for $c_1$ with prescribed
Neumann boundary driving.
The nonlinear nature of $f$ has not yet kicked in at these short timescales.

A numerical solution for $c_1$ could either be done by timestepping
Duhamel ODEs for each above eigenmode, or via an immersed interface finite-difference
solver or time-domain boundary integral solver.
We are unsure of the biological relevance of this inner solution,
especially for the very rapid relaxation times that a spatially-complicated $C_i$ would bring.
%since $C$ is in practice likely to be equilibrated.
Hence we move to the outer solution.



% ooooooooooooooooooooooooooooooooooooooooooooooooooooooooooooooooooooooo
\subsection{Outer layer at zeroth order and uniform approximation}

We return to the original time variable $t$,
and call the zeroth order solution $B_0$ and $C_0$.
Taking $\eps\to 0^+$ replaces \eqref{pde}--\eqref{fick}
by a static homogenous Neumann Laplace BVP at each time $t$,
so that
\bea
\dot{B_0} - \LpO B_0   &=&  f(B_0,C_0)  \qquad \mbox{ on } \pO \times (0,\infty)
\label{surfpde0}
\\
\Delta C_0   &=& 0   \qqquad \mbox{ in } \Omega \times (0,\infty)
\label{pde0}
\\
\dn C_0  &=& 0 \qqquad \mbox{ on } \pO \times (0,\infty)
\label{fick0}
\eea
thus $C_0$ is spatially constant at each time $t$.
However setting $\eps=0$ does {\em not} tell you how this
constant changes with $t$, since there is no PDE for $C$ involving time!
For this, curiously, Prop.~\ref{p:cons} (a result for $\eps>0$) is also needed,
giving the spatially-constant
\be
C_0(x,t) = \frac{1}{\vol\Omega}\biggl[ A_0 - \int_\pO B_0(\cdot,t) \biggr]
~, \qquad x\in\Omega
\label{C0}
\ee
where $A_0$ is some fixed total amount of chemical.
Substituting into \eqref{surfpde0} leaves a self-contained surface
PDE for $B_0$ alone,
\be
\dot{B_0} - \LpO B_0 =f \biggl( B_0 , \frac{1}{\vol\Omega} \biggl[ A_0 - \int_\pO B_0(\cdot,t) \biggr]
\biggr)
\label{B0only}
\ee
which is \cite[Eq.~(7)]{diegmiller18} and assumed throughout that paper
(note they use the symbol ``$C_0$'' to denote our $\frac{A_0}{\vol\Omega}$).

As is standard for outer layer solutions,
\eqref{surfpde0}--\eqref{fick0} cannot match arbitrary ICs:
the initial $C_0$ must be constant in space.
Instead the outer ICs (for $t\to 0^+$) are found by matching
to the inner layer (for $\bar{t}\to\infty$),
giving,
somewhat predictably,
$C_0(x,0)=\bar{C}_i$ for $x\in\Omega$, and
$B_0(x,0)=B_i(x)$ for $x\in\pO$.
As expected, $A_0 = \int_\Omega C_i + \int_\pO B_i$ is the total amount
of initial chemical.
The lowest-order uniform asymptotic expansion is then given by adding
inner and outer solutions and subtracting their common limit,
giving
\be
B^{(u)}(x,t) = B_0(x,t), \quad x\in\pO,\;t>0,
\qquad
C^{(u)}(x,t) = C_0(x,t) + \sum_{j=1}^\infty \alpha_j e^{-\lambda_j t/\eps} \psi_j(x),
\quad x\in\Omega, \; t>0.
\label{unif0}
\ee
This has $\bigO(\eps)$ error uniformly in time, formally justifying
the approximation in \cite{diegmiller18}.



\subsection{First order correction to outer layer solution}

Gathering terms in $\eps$, the PDEs and BC for the first order outer correction are
\bea
\dot{B_1} - f_B(B_0,C_0) B_1 - \LpO B_1
&=&  f_C(B_0,C_0) C_1  \qquad \mbox{ on } \pO \times (0,\infty)
\label{surfpde1}
\\
- \Delta C_1   &=& -\dot{C_0}   \qquad\qqquad \mbox{ in } \Omega \times (0,\infty)
\label{pde1}
\\
\dn C_1  &=& -f(B_0,C_0) \qquad \mbox{ on } \pO \times (0,\infty)
\label{fick1}
\eea
Here $f_B$ and $f_C$ are the partials of the fixed function $f$.
The form is a linear surface reaction-diffusion equation for $B_1$,
coupled to a family of static Neumann Poisson BVPs for $C_1$ at each $t>0$.
We leave the ICs open, to be found by matched asymptotics later.

Consider the static Poisson BVP \eqref{pde1}--\eqref{fick1}.
Its volume forcing $-\dot{C_0}$ is constant in space, by \eqref{C0}.
This constant is found by using \eqref{C0},
the surface divergence theorem, then
\eqref{surfpde0},
\be
-(\vol\Omega)\dot{C_0}(x,t) = \int_\pO \dot{B_0}(\cdot, t) =
\int_\pO [\dot{B_0}(\cdot, t) - \LpO B_0(\cdot,t) ]=
\int_\pO f(B_0(\cdot,t),C_0(\cdot,t))~, \quad x\in\Omega.
\label{C0d}
\ee
For a solution to the BVP to exist one must have (by the divergence theorem)
the total volume forcing matching the total outgoing flux,
i.e.
$$
-\int_\Omega \dot{C_0}(\cdot,t) = -\int_\pO \dn C_1(\cdot,t)~,  \quad \forall t>0~.
$$
Yet this follows since both sides are equal to 
$\int_\pO f(B_0(\cdot,t),C_0(\cdot,t))$, by \eqref{C0d} and 
\eqref{fick1} respectively.
Thus the BVP is {\em consistent} at each $t>0$.
It also has a nullspace of constant functions, so which solution
is selected at each $t>0$ ?
As with zeroth order, it is not contained in the PDE at this order,
so it seems we have to bring in
Prop.~\ref{p:cons} again, giving
\be
\int_\Omega C_1(\cdot,t) = -\int_\pO B_1(\cdot,t)~.
\label{C1}
\ee
Suppose that, at a given $t$,
a Poisson solution to \eqref{pde1}--\eqref{fick1} has been found,
$\tilde{C}_1(x,t)$.
Then, writing $C_1 = \tilde{C}_1 + \alpha$,
by \eqref{C1} one gets, in terms of known quantities at the current $t$,
$$
\alpha = -\frac{1}{\vol\Omega} \biggl[
  \int_\Omega \tilde{C}_1 + \int_\pO B_1
  \biggr]
$$
and then one substitutes this into \eqref{surfpde1} to give
the right-hand side for the surface PDE to timestep,
$$
\dot{B_1} - f_B(B_0,C_0) B_1 - \LpO B_1
=  f_C(B_0,C_0) (\tilde{C}_1 + \alpha)
~.
$$
Together, the last two equations (which could be combined by eliminating $\alpha$)
comprise the first-order equivalent of \eqref{B0only}.

A numerical solution of the Poisson BVP at this order is
quite easy, since a quadratic particular solution can be written down
(due to the spatially-constant forcing), then the boundary Neumann Laplace
problem solved via a 2nd-kind BIE.
A numerical scheme for evolving $B_1$ alongside $B_0$,
and $C_1$ alongside $C_0$, is thus clear,
at the effort of doubling the surface variables, and a Laplace Neumann
BVP solve per timestep.



%The solution $b=b_0+\eps b_1$ and $c=c_0+\eps c_1$ is already $\bigO(\eps^2)
%\sim 10^{-5}$
%accurate, so there is little point in higher-order solutions.

\section{Conclusions}

Using an asymptotic expansion in $\eps$ (essentially the reciprocal bulk diffusivity),
we wrote an approximate solution to the nonlinear coupled
interior-boundary cell diffusion IBVP that is uniformly accurate
to $\bigO(\eps)$, using zeroth-order inner and outer solutions.
This formally justifies the quasistatic bulk
approximation in \cite{diegmiller18}.
The inner layer is biologically irrelevant.
We also write the first-order outer solution.
The outer solution (at each order) has a curious extra condition on the change in average value due to the Neumann boundary condition for the coupling.

Numerically, a scheme for the outer solution
is to solve the zeroth order (i.e. solve for $B_0$ and $C_0$),
then first order (i.e. solve for $B_1$ and $C_1$, which could be
done at the same time as zeroth order, at the cost of one Laplace BVP
solve per timestep),
then add: $B_0+\eps B_1$ and $C_0 + \eps C_1$ is an outer solution
with error $\bigO(\eps^2)$.

However, we believe a cleaner and higher-order numerical scheme could be found;
see other notes.
This note remains mostly a theoretical foundation.

Future directions:
\bi
\item Matched asymptotics including all $\bigO(\eps)$ terms,
  to give an approximate solution with uniform error $\bigO(\eps^2)$.
  This will involve integrals over $f(b_0,c_0)$ with $b_0$ const in $\bar{t}$,
  but $c_0$ sweeping through prescribed values.
  It is not simple.
  It is unclear how $c_1$ matches to $C_1$.
\item Find a numerical method that doesn't require solving the
  zeroth- and then first-order correction sequentially.
  Explore 2nd-kind Volterra time-domain BIE here, although the
  $\eps$-expansion of heat potentials is proving to be a mess.
\ei

  


% BBBBBBBBBBBBBBBBBBBBBBBBBBBBBBBBBBBBBBBBBBBBBBBBBBBBBBBBBBBBBBBBBBBBBBBBBBBB
\bibliographystyle{abbrv}
\bibliography{localrefs}
\end{document}

