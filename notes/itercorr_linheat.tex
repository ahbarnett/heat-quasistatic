\documentclass[10pt]{article}
\oddsidemargin = 0.2in
\topmargin = -0.5in
\textwidth 6in
\textheight 8.5in

\usepackage{graphicx,bm,hyperref,amssymb,amsmath,amsthm}

% -------------------------------------- macros --------------------------
% general ...
\newcommand{\bi}{\begin{itemize}}
\newcommand{\ei}{\end{itemize}}
\newcommand{\ben}{\begin{enumerate}}
\newcommand{\een}{\end{enumerate}}
\newcommand{\be}{\begin{equation}}
\newcommand{\ee}{\end{equation}}
\newcommand{\bea}{\begin{eqnarray}} 
\newcommand{\eea}{\end{eqnarray}}
\newcommand{\ba}{\begin{align}} 
\newcommand{\ea}{\end{align}}
\newcommand{\bse}{\begin{subequations}} 
\newcommand{\ese}{\end{subequations}}
\newcommand{\bc}{\begin{center}}
\newcommand{\ec}{\end{center}}
\newcommand{\bfi}{\begin{figure}}
\newcommand{\efi}{\end{figure}}
\newcommand{\ca}[2]{\caption{#1 \label{#2}}}
\newcommand{\ig}[2]{\includegraphics[#1]{#2}}
\newcommand{\bmp}[1]{\begin{minipage}{#1}}
\newcommand{\emp}{\end{minipage}}
\newcommand{\pig}[2]{\bmp{#1}\includegraphics[width=#1]{#2}\emp} % mp-fig, nogap
\newcommand{\bp}{\begin{proof}}
\newcommand{\ep}{\end{proof}}
\newcommand{\ie}{{\it i.e.\ }}
\newcommand{\eg}{{\it e.g.\ }}
\newcommand{\etal}{{\it et al.\ }}
\newcommand{\pd}[2]{\frac{\partial #1}{\partial #2}}
\newcommand{\pdc}[3]{\left. \frac{\partial #1}{\partial #2}\right|_{#3}}
\newcommand{\infint}{\int_{-\infty}^{\infty} \!\!}      % infinite integral
\newcommand{\tbox}[1]{{\mbox{\tiny #1}}}
\newcommand{\mbf}[1]{{\mathbf #1}}
\newcommand{\half}{\mbox{\small $\frac{1}{2}$}}
\newcommand{\C}{\mathbb{C}}
\newcommand{\N}{\mathbb{N}}
\newcommand{\R}{\mathbb{R}}
\newcommand{\Z}{\mathbb{Z}}
\newcommand{\RR}{\mathbb{R}^2}
\newcommand{\ve}[4]{\left[\begin{array}{r}#1\\#2\\#3\\#4\end{array}\right]}  % 4-col-vec
\newcommand{\vt}[2]{\left[\begin{array}{r}#1\\#2\end{array}\right]} % 2-col-vec
\newcommand{\bigO}{{\mathcal O}}
\newcommand{\qqquad}{\qquad\qquad}
\newcommand{\qqqquad}{\qqquad\qqquad}
\DeclareMathOperator{\Span}{Span}
\DeclareMathOperator{\im}{Im}
\DeclareMathOperator{\re}{Re}
\DeclareMathOperator{\vol}{vol}
\newtheorem{thm}{Theorem}
\newtheorem{cnj}[thm]{Conjecture}
\newtheorem{lem}[thm]{Lemma}
\newtheorem{cor}[thm]{Corollary}
\newtheorem{pro}[thm]{Proposition}
\newtheorem{rmk}[thm]{Remark}
% this work...
\newcommand{\pO}{{\partial\Omega}}
\newcommand{\LpO}{\Delta_\pO}
\newcommand{\eps}{\epsilon}
\newcommand{\dn}{\partial_n}


\begin{document}

\title{Iterative correction in powers of reciprocal diffusivity for linear heat initial boundary value problems}

\author{Alex H. Barnett}
\date{\today}
\maketitle

\begin{abstract}
  Motivated by a cell polarization model of Diegmiller et al (2018)
  considered in another set of notes,
  we consider simpler linear IBVPs for the heat equation,
  in the relevant case of large diffusivity $\kappa$.
  We sketch a numerical boundary integral method that achieves
  arbitrary order accuracy in $\eps = 1\/kappa \ll 1$.
  This replaces history-dependent heat potentials in favor of a
  the quasi-static solution plus a sequence of corrections each involving
  one local derivative in time and one static Poisson solve.
  We show that initial data is irrelevant for times beyond $\bigO(\eps)$.
  Neglecting discretization errors, the 
  result matches boundary data exactly but does not exactly solve the heat
  equation.
  The latter residual is uniformly $\bigO(\eps^p)$
  if $p$ correction steps are done, i.e. it gains one order of $\eps$
  per correction step.
  Some stability results on IBVP solutions with respect to volume driving
  are thus needed to bound the solution error.

  We consider the Dirichlet and (trickier) Neumann cases.
\end{abstract}

\section{Introduction}

We are motivated by a coupled cell polarization problem \cite{diegmiller18}
where the interior heat equation has Neumann boundary conditions
(a net flux) given as a nonlinear function of the local interior and surface concentrations.
Here the bulk diffusivity is at least $10^2$ times larger than the surface diffusivity, giving in interesting quasi-static limit.
A numerical quasistatic approximate solver exists which
requires surface diffusion alone.
To achieve higher accuracy, in general shapes in 3D,
possibility would be a full heat-equation solver; this would require
a lot of technology, including short-time quadratures and modal history compression in the style of Greengard, Strain, Li, Wang, etc.
We present a simpler alternative appropriate for the case of large bulk
diffusivity.
We start with simpler linear IBVPs to present the idea.


\section{The Dirichlet case}

Given a fixed inverse diffusivity $\eps\ll 1$, and final time $T>0$,
we wish to efficiently solve the IBVP
\bea
\eps\dot{u} - \Delta u &=& 0    \qquad \mbox{ in } \Omega\times (0,T)
\label{pde}
\\
u  &=& f  \qquad \mbox{ on } \pO \times (0,T)
\label{bc}
\\
u(\cdot, 0)    &=&  w \qquad \mbox{ in } \Omega
\label{ic}
\eea
where dot is $\partial/\partial t$, and the given
time-dependent boundary data is $f(x,t)$, and initial data $w(x)$.






% BBBBBBBBBBBBBBBBBBBBBBBBBBBBBBBBBBBBBBBBBBBBBBBBBBBBBBBBBBBBBBBBBBBBBBBBBBBB
\bibliographystyle{abbrv}
\bibliography{localrefs}
\end{document}

