\documentclass[10pt]{article}
\oddsidemargin = 0.2in
\topmargin = -0.5in
\textwidth 6in
\textheight 8.5in

\usepackage{graphicx,bm,hyperref,amssymb,amsmath,amsthm}

% -------------------------------------- macros --------------------------
% general ...
\newcommand{\bi}{\begin{itemize}}
\newcommand{\ei}{\end{itemize}}
\newcommand{\ben}{\begin{enumerate}}
\newcommand{\een}{\end{enumerate}}
\newcommand{\be}{\begin{equation}}
\newcommand{\ee}{\end{equation}}
\newcommand{\bea}{\begin{eqnarray}} 
\newcommand{\eea}{\end{eqnarray}}
\newcommand{\ba}{\begin{align}} 
\newcommand{\ea}{\end{align}}
\newcommand{\bse}{\begin{subequations}} 
\newcommand{\ese}{\end{subequations}}
\newcommand{\bc}{\begin{center}}
\newcommand{\ec}{\end{center}}
\newcommand{\bfi}{\begin{figure}}
\newcommand{\efi}{\end{figure}}
\newcommand{\ca}[2]{\caption{#1 \label{#2}}}
\newcommand{\ig}[2]{\includegraphics[#1]{#2}}
\newcommand{\bmp}[1]{\begin{minipage}{#1}}
\newcommand{\emp}{\end{minipage}}
\newcommand{\pig}[2]{\bmp{#1}\includegraphics[width=#1]{#2}\emp} % mp-fig, nogap
\newcommand{\bp}{\begin{proof}}
\newcommand{\ep}{\end{proof}}
\newcommand{\ie}{{\it i.e.\ }}
\newcommand{\eg}{{\it e.g.\ }}
\newcommand{\etal}{{\it et al.\ }}
\newcommand{\pd}[2]{\frac{\partial #1}{\partial #2}}
\newcommand{\pdc}[3]{\left. \frac{\partial #1}{\partial #2}\right|_{#3}}
\newcommand{\infint}{\int_{-\infty}^{\infty} \!\!}      % infinite integral
\newcommand{\tbox}[1]{{\mbox{\tiny #1}}}
\newcommand{\mbf}[1]{{\mathbf #1}}
\newcommand{\half}{\mbox{\small $\frac{1}{2}$}}
\newcommand{\C}{\mathbb{C}}
\newcommand{\N}{\mathbb{N}}
\newcommand{\R}{\mathbb{R}}
\newcommand{\Z}{\mathbb{Z}}
\newcommand{\RR}{\mathbb{R}^2}
\newcommand{\ve}[4]{\left[\begin{array}{r}#1\\#2\\#3\\#4\end{array}\right]}  % 4-col-vec
\newcommand{\vt}[2]{\left[\begin{array}{r}#1\\#2\end{array}\right]} % 2-col-vec
\newcommand{\bigO}{{\mathcal O}}
\newcommand{\qqquad}{\qquad\qquad}
\newcommand{\qqqquad}{\qqquad\qqquad}
\DeclareMathOperator{\Span}{Span}
\DeclareMathOperator{\im}{Im}
\DeclareMathOperator{\re}{Re}
\DeclareMathOperator{\vol}{vol}
\newtheorem{thm}{Theorem}
\newtheorem{cnj}[thm]{Conjecture}
\newtheorem{lem}[thm]{Lemma}
\newtheorem{cor}[thm]{Corollary}
\newtheorem{pro}[thm]{Proposition}
\newtheorem{rmk}[thm]{Remark}
% this work...
\newcommand{\pO}{{\partial\Omega}}
\newcommand{\LpO}{\Delta_\pO}
\newcommand{\eps}{\epsilon}
\newcommand{\dn}{\partial_n}
\newcommand{\dt}{\partial_t}
\newcommand{\LTO}{{L^2(\Omega)}}
\DeclareMathOperator{\Lap}{Lap}



\begin{document}

\title{Iterative correction in powers of reciprocal diffusivity for heat initial boundary value problems}

\author{Alex H. Barnett}
\date{\today}
\maketitle

\begin{abstract}
  Motivated by a cell polarization model of Diegmiller et al (2018)
  considered in another set of notes,
  we consider simpler linear IBVPs for the heat equation,
  in the relevant case of large diffusivity $\kappa$.
  We sketch a numerical boundary integral method that achieves
  arbitrary order accuracy in $\eps = 1/\kappa \ll 1$.
  This replaces history-dependent heat potentials in favor of a
  the quasi-static solution plus a sequence of corrections each involving
  one local derivative in time and one static Poisson solve.
  We show that initial data is irrelevant for times beyond $\bigO(\eps)$.
  Neglecting discretization errors, the 
  result matches boundary data exactly but does not exactly solve the PDE.
  The latter's residual is uniformly $\bigO(\eps^p)$
  if $p-1$ correction steps are done, i.e.\ it gains one order of $\eps$
  per correction step.
  Some stability results on IBVP solutions with respect to volume driving
  are thus needed to convert this to a bound on the solution error.

  We consider the Dirichlet and (trickier) Neumann cases,
  the latter being closer to the cell model.
\end{abstract}

\section{Introduction}

We are motivated by a coupled cell polarization problem \cite{diegmiller18}
where the interior heat equation has Neumann boundary conditions
(a net flux) given as a nonlinear function of the local interior and surface concentrations.
Here the bulk diffusivity is at least $10^2$ times larger than the surface diffusivity, allowing a
numerical quasistatic approximate solution which
requires surface diffusion alone (see other notes and \cite{diegmiller18}).
Yet to achieve higher accuracy, in general shapes in 3D,
a possibility would be a full heat-equation solver; this would require
a lot of technology, including short-time surface quadratures and modal history compression in the style of Greengard, Strain, Li, Wang, etc.
We present a simpler alternative, which trades time-history for
a volumetric spatial grid for static spatial solves, achieving high-order
accuracy for the case of large bulk diffusivity.
We use simple linear IBVPs to present the idea.


% DDDDDDDDDDDDDDDDDDDDDDDDDDDDDDDDDDDDDDDDDDDDDDDDDDDDDDDDDDDDDDDDDDDDDDDDD
\section{The case of time-dependent Dirichlet data}

Given a fixed inverse diffusivity $\eps\ll 1$,
and final time $T$,
we wish to efficiently solve the IBVP
\bea
\eps\dot{u} - \Delta u &=& 0    \qquad \mbox{ in } \Omega\times (0,T)
\label{pde}
\\
u  &=& f  \qquad \mbox{ on } \pO \times (0,T)
\label{bc}
\\
u(\cdot, 0)    &=&  0 \qquad \mbox{ in } \Omega
\label{ic}
\eea
where dot denotes $\partial/\partial t$, and the given
time-dependent boundary data is $f(x,t)$,
with the usual compatibility $f=0$ on  $\pO\times\{t=0\}$.
It is assumed that $f$ is smooth on the unit timescale, in particular
being oblivious to $\eps$.

Non-zero initial data is not needed here,
because the effect of any initial data decays away
in time $\bigO(\eps)$ (see other notes on asymptotic analysis).
This can be seen by writing $\phi_j$ as the orthonormal eigenfunctions
of $-\Delta$ in $\Omega$ with Dirichlet BCs, and $\lambda_j$
the coresponding eigenvalues, and noting that
$$
u(x,t) = \sum_{j=1}^\infty c_j e^{-\lambda_jt/\eps} \phi_j(x)
$$
satisfies \eqref{pde} with zero BCs and initial data $u(\cdot,0)=w$,
for coefficients $c_j = \langle \phi_j, w \rangle$.

Our approximate method will solve \eqref{bc}--\eqref{ic}, but
not solve \eqref{pde} exactly.
We need to know how that residual controls the overall solution error.

\begin{lem}[Stability with respect to volume driving]
  Let $u$ solve the inhomogeneous IBVP
\bea
\eps\dot{u} - \Delta u &=& g    \qquad \mbox{ in } \Omega\times (0,T)
\label{pdeg}
\\
u  &=& 0  \qquad \mbox{ on } \pO \times (0,T)
\label{bc0}
\\
u(\cdot, 0)    &=&  0 \qquad \mbox{ in } \Omega
\eea
where $\|g(\cdot,t)\|_\LTO \le M$ for all $t\in(0,T)$.
Then
\be
\|u(\cdot,t)\|_\LTO \;\le\;
\frac{1}{\lambda_1}\sup_{t\in(0,T)}\|g(\cdot,t)\|_\LTO~.
\label{gstab}
\ee
\label{l:gstab}
\end{lem}
\begin{proof}
We use the energy method.
Writing $L(t):=\|u(\cdot,t)\|_\LTO$, then by \eqref{pdeg},
$$
\frac{\eps}{2}\dt(L^2) = \eps\int_\Omega u \dot u = \int_\Omega
u\Delta u + \int_\Omega g u
$$
The eigenfunction expansion
$u(\cdot,t) = \sum_{j=1}^\infty a_j \phi_j$
bounds $\int_\Omega u\Delta u = -\sum_{j=1}^\infty \lambda_1 a_j^2
\le -\lambda_1 L^2$.
Applying this, and Cauchy--Schwartz to the $gu$ term, with $M:=\sup_{t\in(0,T)}\|g(\cdot,t)\|_\LTO$,
we get
$$
\eps L\dot L \le -\lambda_1 L^2 + ML~,
$$
with initial condition $L(0)=0$.
Thus $L(t)$ is upper-bounded by the solution to the ODE equality,
which is (by cancelling $L$),
$$
L(t) = \frac{M}{\lambda_1} (1 - e^{-\lambda_1 t / \eps})~.
$$
This never exceeds $M/\lambda_1$.
\end{proof}

Note that the right-hand side of \eqref{gstab} doesn't involve a factor $1/\eps$,
which is surprising because physically the driving
(power density input)
in \eqref{pdeg} is $g/\eps$, i.e., large.
However, the diffusivity $1/\eps$ and Dirichlet BCs suck
the heat out on a rapid timescale $\bigO(\eps)$, so that it doesn't have time to build up in the interior.
For unit-sized domains $\Omega$, $\lambda_1=\bigO(1)$, and cannot be
small by the Faber--Krahn inequality.

\subsection{The idea of iterative correction via spatial solves}

Now to the method for the IBVP \eqref{pde}--\eqref{ic}.
Let $u_1$ solve the quasistatic Laplace problem which sets $\eps=0$
in \eqref{pde}.
Its residual is thus $(\eps\dt-\Delta)u_1 = \eps \dot u_1 =: \eps r_1$.
So, applying Lemma~\ref{l:gstab} to $u_1-u$
shows that $u_1$ is an $\bigO(\eps)$ accurate solution
to the IBVP.
Its cost is one Laplace solve per timestep (and since no evolution
equations are solved, the timestep is dictated only by the user's desired
evaluation time grid).
It is the Dirichlet equivalent of the quasistatic approximation
used in \cite{diegmiller18}.

Let $v_1$ solve the static Poisson equation $\Delta v_1 = r_1$ with
BC $v_1=0$ on $\pO$. Recall that $r_1=\dot u_1$ is a function
on $\Omega \times (0,T)$, so a fresh Poisson solve is needed for each time $t$.
Consider the corrected trial function
$u_2 := u_1 + \eps v_1$. Its residual cancels by construction as follows,
$$
(\eps\dt-\Delta)(u_1 +\eps v_1) = \eps r_1 - \eps \Delta v_1 +\eps^2 \dot v_1 =
\eps^2 \dot v_1 =: \eps^2 r_2~.
$$
Applying Lemma~\ref{l:gstab} to $u_2-u$
shows that $u_2$ is an $\bigO(\eps^2)$ accurate solution
to the IBVP. Its additional cost over $u_1$ is one Poisson solve
and one extra Laplace solve per timestep.
It would get around 5-digit accuracy for the
$\eps\approx 0.003$ in the cell polarization application in \cite{diegmiller18},
which is probably adequate.

One may iterate the above, first taking a time-derivative $r_{k} = \dot v_{k-1}$,
then doing a Poisson solve $\Delta v_k = r_k$
with $v_k=0$ on $\pO$, for $k=1,\dots,p-1$,
then summing
\be
u_p := u_1 + \eps v_1 + \eps^2 v_2 + \dots + \eps^{p-1} v_{p-1}
\label{up}
\ee
which has residual $(\eps\dt-\Delta)u_p = \eps^p \dot v_{p-1}$,
and thus by the Lemma is an $\bigO(\eps^p)$ accurate solution
to the IBVP.
However, all correction operations are local to the current time $t$.
The total cost is $p-1$ Poisson solves and $p$ Laplace solves per timestep.
Choosing even $p=4$ would get around 10-digit accuracy for the $\eps$
mentioned above.

Let $\Lap_D: C(\pO) \to C(\Omega)$ be the solution operator for the Dirichlet
Laplace equation in $\Omega$, and let $\Delta_0^{-1} : C^k(\Omega) \to
C^{k+2}(\Omega)$ be the solution operator for the Poisson equation with
zero Dirichlet boundary data. When acting on functions of
space and time, each operator is taken to act locally in time
(i.e., independently on time slices).
Then another, formal, way to write \eqref{up} is
\bea
u_p &=& u_1 + \eps \Delta_0^{-1} \dot u_1 + \eps^2 \Delta_0^{-2} \ddot u_1 +
\dots + \eps^{p-1} \Delta_0^{-(p-1)} \dt^{p-1} u_1
\\
&=&
\Lap_D f + \eps \Delta_0^{-1} \Lap_D \dot f + \ldots
+ \eps^{p-1} \Delta_0^{-(p-1)} \Lap_D f^{(p-1)}
~.
\label{upformal}
\eea



\subsection{Implementation notes}


*** to do.

$n^3$ interior points. Can avoid?

Each operator in \eqref{upformal} is a routine spatial solve, or a pointwise
time-derivative.

$\Lap_D(f(\cdot,t))$ may be solved via a 
a boundary integral equation method on $\pO$.
To get $\Lap_D(\dt^kf(\cdot,t))$ for $k=1,\dots,p-1$,
since $\dt$ commutes with $\Lap_D$, a set of $p$ Laplace solutions
$\Lap_D(f(\cdot,t+\tau_j))$ for a local time grid $\{\tau_j\}$
could be computed.

$\Delta_0^{-1} r$ may be solved via a 
a box-code FMM plus the above boundary integral
equation method, inputting and outputting on volumetric quadrature
points. Ideally this set of points would be the same.

For small problems with a fixed geometry, the boundary solution operators
could be prestored either in dense or compressed form.

Problems of interest would modify IBVP so that $f$ is computed on the
fly in terms of surface variables and surface limits of $u$ at the current time.
In this case a BDF formula could be used to extract $\dt^k f$ for $k=1,\dots,p-1$.


% NNNNNNNNNNNNNNNNNNNNNNNNNNNNNNNNNNNNNNNNNNNNNNNNNNNNNNNNNNNNNNNNNNNNNNNNNNN
\section{The Neumann case}

\bea
\eps\dot{u} - \Delta u &=& 0    \qquad \mbox{ in } \Omega\times (0,T)
\label{pden}
\\
\dn u  &=& f  \qquad \mbox{ on } \pO \times (0,T)
\label{bcn}
\\
u(\cdot, 0)    &=&  0 \qquad \mbox{ in } \Omega
\eea

*** is the Laplace consistency cond on $f$ enforced at each time?
Otherwise $u$ shoots up at speed $1/\eps$ which is huge, and
a smooth solution on unit timescales is impossible.
In the cell application, consistency applies to order $\eps$.
A weird set-up.

The zero mode wreaks havoc.
*** to do.







\section{Discussion}

\bi
\item
  Clearly finite-difference schemes with explicit timestepping are
  a disaster, since $\delta t \le c \eps h^2$.
  Implicit is worth exploring; in fact higher-order
  implicit schemes may be related to the correction iteration \eqref{up}.
\item
  Taking arbitrarily high time derivatives of the BC data $f$ is dangerous
  and loses digits. A heat-potential scheme may not lose these digits,
  but is much more complicated to get correct.
\item
Strangely, if $f(x,\cdot) \in C^\infty((0,T))$,
taking $p\to\infty$ in \eqref{upformal} gives the formal Neumann series
\be
u_\infty := Lap_D f + \eps \Delta_0^{-1} \dt \Lap_D f + \eps^2 \Delta_0^{-2} \dt^2 \Lap_D f + \dots
= (I - \eps \Delta_0^{-1} \dt)^{-1} \Lap_D f
\ee
whose meaning is obscure.
It is not a mere translation in time since it has no
reciprocal factorial factors, making its convergence a mystery
even for small $eps$.
\item
  I have tried analogous $\eps$-expansions for heat potentials
  to solve the Dirichlet IBVP.
  However, the mass of the algebraic tails in time decays like
  fractional powers of $\eps$, and it is not obvious how to
  write the result as a perturbation from the quasistatic solution $u_1$.
  I am far from a numerical method there. In contrast the above seems
  quite simple.
  \ei


% BBBBBBBBBBBBBBBBBBBBBBBBBBBBBBBBBBBBBBBBBBBBBBBBBBBBBBBBBBBBBBBBBBBBBBBBBBBB
\bibliographystyle{abbrv}
\bibliography{localrefs}
\end{document}

