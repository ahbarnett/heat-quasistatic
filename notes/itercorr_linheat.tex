\documentclass[10pt]{article}
\oddsidemargin = 0.2in
\topmargin = -0.5in
\textwidth 6in
\textheight 8.5in

\usepackage{graphicx,bm,hyperref,amssymb,amsmath,amsthm}
\usepackage{algorithmic,xcolor}

% -------------------------------------- macros --------------------------
% general ...
\newcommand{\bi}{\begin{itemize}}
\newcommand{\ei}{\end{itemize}}
\newcommand{\ben}{\begin{enumerate}}
\newcommand{\een}{\end{enumerate}}
\newcommand{\be}{\begin{equation}}
\newcommand{\ee}{\end{equation}}
\newcommand{\bea}{\begin{eqnarray}} 
\newcommand{\eea}{\end{eqnarray}}
\newcommand{\ba}{\begin{align}} 
\newcommand{\ea}{\end{align}}
\newcommand{\bse}{\begin{subequations}} 
\newcommand{\ese}{\end{subequations}}
\newcommand{\bc}{\begin{center}}
\newcommand{\ec}{\end{center}}
\newcommand{\bfi}{\begin{figure}}
\newcommand{\efi}{\end{figure}}
\newcommand{\ca}[2]{\caption{#1 \label{#2}}}
\newcommand{\ig}[2]{\includegraphics[#1]{#2}}
\newcommand{\bmp}[1]{\begin{minipage}{#1}}
\newcommand{\emp}{\end{minipage}}
\newcommand{\pig}[2]{\bmp{#1}\includegraphics[width=#1]{#2}\emp} % mp-fig, nogap
\newcommand{\bp}{\begin{proof}}
\newcommand{\ep}{\end{proof}}
\newcommand{\ie}{{\it i.e.\ }}
\newcommand{\eg}{{\it e.g.\ }}
\newcommand{\etal}{{\it et al.\ }}
\newcommand{\pd}[2]{\frac{\partial #1}{\partial #2}}
\newcommand{\pdc}[3]{\left. \frac{\partial #1}{\partial #2}\right|_{#3}}
\newcommand{\infint}{\int_{-\infty}^{\infty} \!\!}      % infinite integral
\newcommand{\tbox}[1]{{\mbox{\tiny #1}}}
\newcommand{\mbf}[1]{{\mathbf #1}}
\newcommand{\half}{\mbox{\small $\frac{1}{2}$}}
\newcommand{\C}{\mathbb{C}}
\newcommand{\N}{\mathbb{N}}
\newcommand{\R}{\mathbb{R}}
\newcommand{\Z}{\mathbb{Z}}
\newcommand{\RR}{\mathbb{R}^2}
\newcommand{\ve}[4]{\left[\begin{array}{r}#1\\#2\\#3\\#4\end{array}\right]}  % 4-col-vec
\newcommand{\vt}[2]{\left[\begin{array}{r}#1\\#2\end{array}\right]} % 2-col-vec
\newcommand{\bigO}{{\mathcal O}}
\newcommand{\qqquad}{\qquad\qquad}
\newcommand{\qqqquad}{\qqquad\qqquad}
\DeclareMathOperator{\Span}{Span}
\DeclareMathOperator{\im}{Im}
\DeclareMathOperator{\re}{Re}
\DeclareMathOperator{\vol}{vol}
\newtheorem{thm}{Theorem}
\newtheorem{cnj}[thm]{Conjecture}
\newtheorem{lem}[thm]{Lemma}
\newtheorem{cor}[thm]{Corollary}
\newtheorem{pro}[thm]{Proposition}
\newtheorem{rmk}[thm]{Remark}
% this work...
\newcommand{\pO}{{\partial\Omega}}
\newcommand{\LpO}{\Delta_\pO}
\newcommand{\eps}{\epsilon}
\newcommand{\dn}{\partial_n}
\newcommand{\dt}{\partial_t}
\newcommand{\LTO}{{L^2(\Omega)}}
\newcommand{\LTpO}{{L^2(\pO)}}
\DeclareMathOperator{\Lap}{Lap}



\begin{document}

\title{Quasistatic correction in powers of reciprocal diffusivity for heat initial boundary value problems}
% asymptotic expansion?


\author{Alex H. Barnett}
\date{\today}
\maketitle

\begin{abstract}
  Motivated by a cell polarization model of Diegmiller et al (2018),
  we consider IBVPs for the heat equation
  in the case of large diffusivity $\kappa$.
  We sketch a numerical asymptotic method
  that achieves arbitrary order accuracy in $\eps = 1/\kappa \ll 1$,
  exploiting the fact that in this case the heat equation has a short memory
  with timescale $\bigO(\eps)$.
  This replaces history-dependent heat potentials in favor of a
  quasi-static solution plus an asymptotic series of corrections
  involving time-derivatives and {\em elliptic} solves.
  Each such correction buys an extra order of $\eps$ accuracy.
%  We show that initial data is irrelevant for times beyond $\bigO(\eps)$.
%  Neglecting discretization errors, the 
%  result matches boundary data exactly but does not exactly solve the PDE.
%  The latter's residual is uniformly $\bigO(\eps^p)$
%  if $p-1$ correction steps are done, i.e.\ it gains one order of $\eps$
%  per correction step.

  We consider both Dirichlet and Neumann linear IBVPs.  The latter has
  peculiarities involving the zero mode, is closely related to the cell
  model, and requires only {\em one Laplace boundary solve per
  timestep} to get provable $\bigO(\eps^2)$ accuracy.
  Boundary integral implementations are sketched, either with or
  without interior discretization.  (A nonlinear coupled IBVP method
  needs to be sketched\ldots)  A rigorous error analysis of the order in $\eps$
  is given for both Dirichlet and Neumann cases,
  combining energy estimates with estimates on Poisson problems.
\end{abstract}

\section{Introduction}

We are motivated by a coupled cell polarization problem \cite{diegmiller18}
where the interior heat equation has Neumann boundary conditions
(a net flux) given as a nonlinear function of the local interior and surface concentrations.
Here the bulk diffusivity is at least $10^2$ times larger than the surface diffusivity, allowing a
numerical quasistatic approximate solution which
requires surface diffusion alone (see other notes and \cite{diegmiller18}).
Yet to achieve higher accuracy, in general shapes in 3D,
a possibility would be a full heat-equation solver; this would require
a lot of technology, including short-time surface quadratures and modal history compression in the style of Greengard, Strain, Li, Wang, etc.
We present a simpler alternative, which trades time-history for
a sequence of static spatial solves,
achieving various asymptotic orders with respect to large bulk diffusivity.

We use simple linear IBVPs to present the idea, for a fixed
bounded smooth domain $\Omega \in \R^d$, in general dimension $d$.
We start with Dirichlet, then give Neumann (which is a linear toy for the
cell polarization problem).
The Neumann has tricker analysis, but ends up getting one order
higher accuracy than the Dirichet case for the same effort.


% DDDDDDDDDDDDDDDDDDDDDDDDDDDDDDDDDDDDDDDDDDDDDDDDDDDDDDDDDDDDDDDDDDDDDDDDD
\section{The case of time-dependent Dirichlet data}

Given a fixed inverse diffusivity $\eps\ll 1$,
and final time $T$,
we wish to efficiently solve the IBVP
\bea
\eps\dot{u} - \Delta u &=& 0    \qquad \mbox{ in } \Omega\times (0,T)
\label{pde}
\\
u  &=& f  \qquad \mbox{ on } \pO \times (0,T)
\label{bc}
\\
u(\cdot, 0)    &=&  0 \qquad \mbox{ in } \Omega
\label{ic}
\eea
where dot denotes $\partial/\partial t$, and the given
time-dependent boundary data is $f(x,t)$,
with the usual compatibility $f=0$ on  $\pO\times\{t=0\}$.
It is assumed that $f$ is smooth on the unit timescale.
It is standard that the IBVP has a unique solution.

Non-zero initial data is not needed here,
because the effect of any initial data decays away
in time $\bigO(\eps)$ (see other notes on asymptotic analysis).
This can be seen by writing $\phi_j$ as the orthonormal eigenfunctions
of $-\Delta$ in $\Omega$ with Dirichlet BCs, and $\lambda_j$
the coresponding eigenvalues, and noting that
$$
u(x,t) = \sum_{j=1}^\infty c_j e^{-\lambda_jt/\eps} \phi_j(x)
$$
satisfies \eqref{pde} with zero BCs and initial data $u(\cdot,0)=w$,
for coefficients $c_j = \langle \phi_j, w \rangle$.

Our asymptotic method will solve \eqref{bc}--\eqref{ic}, but
only solve \eqref{pde} to various orders in $\eps$.
We need to know how this PDE residual controls the overall solution error.

\begin{lem}[Stability with respect to volume driving, Dirichlet case]  % llllllllllllllllllllll
  Let $w$ solve the inhomogeneous IBVP
\bea
\eps\dot{w} - \Delta w &=& \eps g    \qquad \mbox{ in } \Omega\times (0,T)
\label{pdeg}
\\
w  &=& 0  \qquad \mbox{ on } \pO \times (0,T)
\label{bc0}
\\
w(\cdot, 0)    &=&  0 \qquad \mbox{ in } \Omega ~.
\eea
Then, in terms of $\lambda_j$, $j=1,2,\dots$,
the nondecreasing Dirichlet Laplace eigenvalues of $\Omega$,
\be
\|w(\cdot,t)\|_\LTO \;\le\;
\frac{\eps}{\lambda_1}\sup_{t\in(0,T)}\|g(\cdot,t)\|_\LTO~.
\label{gstab}
\ee
\label{l:gstab}
\end{lem}
\begin{proof}
We use an energy method.
Writing $L(t):=\|w(\cdot,t)\|_\LTO$, then using \eqref{pdeg},
\be
\frac{\eps}{2}\dt(L^2) = \eps\int_\Omega w \dot w = \int_\Omega
w\Delta w + \eps \int_\Omega g w
\label{energy}
\ee
The compatible eigenfunction expansion
$w(\cdot,t) = \sum_{j=1}^\infty a_j \phi_j$
bounds $\int_\Omega w\Delta w = -\sum_{j=1}^\infty \lambda_j a_j^2
\le -\lambda_1 L^2$.
Applying this, and Cauchy--Schwartz to the $gw$ term, with $M:=\sup_{t\in(0,T)}\|g(\cdot,t)\|_\LTO$,
we get
$$
\eps L\dot L \le -\lambda_1 L^2 + \eps ML~,
$$
with initial condition $L(0)=0$.
Thus $L(t)$ is upper-bounded by the solution to the ODE equality,
which is (by cancelling $\eps L$),
$$
L(t) = \frac{\eps M}{\lambda_1} (1 - e^{-\lambda_1 t / \eps})~.
$$
This never exceeds $\eps M/\lambda_1$.
\end{proof}

Note that the right-hand side of \eqref{gstab} is $\bigO(\eps)$,
which is surprising because physically the driving
(power density input) in \eqref{pdeg} is $g$, which is $\bigO(1)$.
However, the diffusivity $1/\eps$ and Dirichlet BCs suck
the heat out on a rapid timescale $\bigO(\eps)$, so that it doesn't have time to build up in the interior.
For unit-sized domains $\Omega$, $\lambda_1=\bigO(1)$, and cannot be
small by the Faber--Krahn inequality.


\subsection{Correction to arbitrary order via static solves}

Now to the method for the IBVP \eqref{pde}--\eqref{ic}.
Let $u_0$ solve the quasistatic Laplace problem which sets $\eps=0$
in \eqref{pde}.
Its residual is thus $(\eps\dt-\Delta)u_0 = \eps \dot u_0 =: \eps r_1$.
So, applying Lemma~\ref{l:gstab} to $w= u_0-u$,
noting that $\dot u$ satisfies the Laplace BVP for $\dot f$ thus is controlled
in magnitude by it,
shows that $u_0$ is an $\bigO(\eps)$ accurate solution
to the IBVP.
Its cost is one Laplace solve per timestep (and since no evolution
equations are solved, the timestep is dictated only by the user's desired
evaluation time grid).
It is the Dirichlet equivalent of the quasistatic approximation
used in \cite{diegmiller18}.

Now let $v_1$ solve the static Poisson equation $\Delta v_1 = r_1$ with
BC $v_1=0$ on $\pO$. Recall that $r_1=\dot u_0$ is a known function
on $\Omega \times (0,T)$, so a Poisson solve is needed at each time $t$.
Consider the corrected trial function
$u_1 := u_0 + \eps v_1$. Its residual cancels by construction as follows,
$$
(\eps\dt-\Delta)(u_0 +\eps v_1) = \eps r_1 - \eps \Delta v_1 +\eps^2 \dot v_1 =
\eps^2 \dot v_1 =: \eps^2 r_2~.
$$
Applying Lemma~\ref{l:gstab} to $w = u_1-u$
shows that $u_1$ is an $\bigO(\eps^2)$ accurate solution
to the IBVP.
One way to phrase its extra cost (per timestep) over $u_0$ as
one Poisson solve plus one time-derivative;
we study the actual cost of implementations below.
It formally requires {\em no history information} if $f$, $\dot f$, and
$\ddot f$ are provided at each desired output time $t$.
It would get around 5-digit accuracy for the
$\eps\approx 0.003$ in the cell polarization application in \cite{diegmiller18},
which is probably adequate.

For any target evaluation time $t$, one may iterate the above
to get any higher-order accuracy $p\ge1$. Here's the formal procedure
(see below for implementations, which move time-derivatives to different
places):

\vspace{1ex}
\colorbox[rgb]{0.9,0.9,0.9}{%
\begin{minipage}{0.9\textwidth}
\begin{algorithmic}
  \STATE let $v_0 := u_0$ be given as the quasistatic Laplace solution to $f$
  \FOR{$k=1,\dots,p-1$}
  \STATE take time-derivative of volume function: \; $r_{k} \gets \dot v_{k-1}$
  \STATE solve Poisson equation with this data: \;\;\; $\Delta v_k = r_k$ in $\Omega$, with $v_k=0$ on $\pO$
  \ENDFOR
\end{algorithmic}
\end{minipage}
}%
\vspace{1ex}

Then the trial solution
\be
u_{p-1} := \sum_{k=0}^{p-1} \eps^k v_k
\label{up}
\ee
matches the BC and IC, and has PDE
residual $(\eps\dt-\Delta)u_{p-1} = \eps^p \dot v_{p-1}$,
and thus by Lemma~\ref{l:gstab} is an $\bigO(\eps^p)$ accurate solution
to the IBVP.
All correction operations are local in time.
Choosing even $p=4$ would get around 10-digit accuracy for the $\eps$
mentioned above.

Here are equivalent formulae for the general $p$ case.
Let $\Lap_D: C(\pO) \to C(\Omega)$ be the solution operator for the Dirichlet
Laplace equation in $\Omega$, and let $\Delta_0^{-1} : C^k(\Omega) \to
C^{k+2}(\Omega)$ be the solution operator for the Poisson equation with
zero Dirichlet boundary data. When acting on functions of
space and time, each operator is taken to act locally in time
(i.e., independently on time slices).
Then another formal way to write \eqref{up} is, recalling $u_0=v_0=\Lap_D f$ is
the quasistatic solution,
\bea
u_p &=& v_0 + \eps \Delta_0^{-1} \dot v_0 + \eps^2 \Delta_0^{-2} \ddot v_0 +
\dots + \eps^{p} \Delta_0^{-p} \dt^{p} v_0
\\
&=&
\Lap_D f + \eps \Delta_0^{-1} \Lap_D \dot f + \ldots
+ \eps^{p} \Delta_0^{-p} \Lap_D \dt^{p} f
~.
\label{upformal}
\eea
There are many other ways to write it because $\dt$ and spatial solves
commute.


\subsection{Implementation I: interior discretization (sketch)}

For spatially simpler problems,
keeping in mind $d=2$ or $d=3$ dimensional applications,
it is practical to fix a
volume quadrature scheme over $\Omega$ with $\bigO(n^d)$ nodes,
$n$ being the number of nodes per linear dimension.
$\pO$ is discretized with $\bigO(n^{d-1})$ nodes.
For simplicity imagine the boundary data $f(\cdot,t_j)$ is supplied on the boundary nodes on a regular time grid $t_j=j\delta$, $j\in\Z$, $\delta>0$.
We will not need $f$'s derivatives to be supplied.
Pick $p$, the desired accuracy order in $\eps$.
We wish to evaluate the approximate solution on the same time grid.

Our solvers are (we ignore close-evaluation aspects for now):
\bi
\item
  For $\Lap_D$, a boundary-integral (BIE) solver followed by FMM evaluation on
  interior nodes. Cost $\bigO(n^d)$, dominated by interior evaluation.
\item
  For $\Delta_0^{-1}$, FMM evaluation from interior to boundary,
  then correction by a BIE, followed by box-code plus boundary density
  evaluation back to interior.
  Cost $\bigO(n^d)$, dominated by interior-to-interior box-code, I guess.
\ei

The algorithm exploits that the time-derivatives can be pushed to the end:
\ben
\item
  For each timestep $j$, fill interior data $w_0(\cdot,t_j)=\Lap_D f(\cdot,t_j)$,
  then for $k=1,\ldots,p-1$,
  iterate $w_k(\cdot,t_j)=\Delta_0^{-1} w_{k-1}(\cdot,t_j)$, storing them all
  on interior nodes.
\item
  Apply a set of finite-difference rules (eg from reading off the lowest $p$ derivatives of a local Lagrange interpolation)
  in the time direction, pointwise for each interior node,
  to compute
  $v_k(\cdot,t_j) = \dt^k w_k(\cdot,t_j)$, for $k=0,\ldots,p-1$.
\item
  Sum the $v_k$ according to \eqref{up} at each node at each timestep.
\een

The result is the interior solution on the time grid at all interior nodes.
Multiple $\eps$ choices could be spat out in this way for free
(in the linear case presented).
Boundary normal-derivative
data could be spat out of each Poisson solve too, but the interior
values are actually needed in all but the last Poisson solve.
The cost is $\bigO(pn^d)$ (dominated by $p-1$ Poisson solves)
per timestep.
Practically, one will sweep through the timesteps in order.
Since the FD stencil width is of order $p$, at least,
to read out $p-1$ derivatives, one needs storage $p^2n^d$,
ie about $p^2$ copies of the volume grid.
So, $p$ cannot be large.
If a marching scheme (for a nonlinear coupled problem) were in play
that generated the data $f$ on the fly,
a BDF would be needed for step 2 above.

Note that the polyharmonic solves evident in \eqref{upformal}
are captured by the repeated Poisson solves. The possible $p^2$ cost is
thus avoided.

Note that digit loss is expected with higher time-derivatives,
but they are hit by increasing powers of $\eps$, so, I imagine this
is not a problem.
I have not thought about the trade-off between order of timestep $\delta$
accuracy vs $\eps$-correction-order.

{\bf Simplest correction: $p=2$.}
Getting $\bigO(\eps^2)$ accuracy needs one Laplace solve
and one Poisson solve per timestep, and a FD (or BDF) formula
to read off $v_1 = \dt w_1$ at all interior nodes.
If the boundary Neumann data (and no interior solution) is needed,
the box-code becomes a plain FMM from interior to boundary
(apart from near-boundary corrections), which will be much faster than
an interior-interior box code.
This is quite a reasonable scheme, basically two
BIE solves plus two interior-to-boundary FMMs per timestep.



\subsection{Implementation II: all on the boundary (sketch)}

Since the heat equation is homogeneous, only boundary values
(specifically, the complementary data, which is Neumann here)
are needed to know the interaction with the surface in a practical
nonlinear coupled problem.
We can entirely avoid interior nodes,
at a cost of polyharmonic BIE solves.

{\bf Simplest correction: $p=2$.}
Recall $v_0$ solves, at each timestep, $\Delta v_0 = f$ with $v_0=0$ on $\pO$.
Thus $\dn v_0$ may be computed by a 2nd-kind BIE solver, e.g.\ for large problems in $\bigO(n^{d-1})$ effort via FMM+GMRES.
Now $v_1$ solves, at each timestep, the biharmonic equation
\bea
\Delta^2 v_1 &=& 0 \qquad \mbox{ in } \Omega
\\
\Delta v_1 &=& \dot f \qquad \mbox{ on } \pO
\\
v_1 &=& 0 \qquad \mbox{ on } \pO~.
\eea
This boundary condition is of inhomogeneous Navier type.\footnote{In the 1D case this is physically a rod mounted at points with sprung pivots transmitting given torques at these points, and zero interior loading.}
Thus $\dn v_1$ may be also computed by a 2nd-kind BIE solver, e.g.\ in $\bigO(n^{d-1})$ effort via FMM plus an iterative solver, but the biharmonic formulation and FMM are more complicated.
For small problems with a fixed geometry, the boundary solution operators
could be prestored either in dense ($n^{d-1} \le 10^4$)
or rank-structured compressed
($n^{d-1} \le 10^6$) form.
The boundary solution $\dn u_1 := \dn v_0 + \eps \dn v_1$ is then $\bigO(\eps^2)$
accurate.

{\bf Higher-order corrections.}
Higher $p$ requires polyharmonic BVP solutions up to order $\Delta^p$.
Namely, $v_k$ solves, at each timestep,
\bea
\Delta^k v_k &=& 0 \qquad \mbox{ in } \Omega
\\
\Delta^{k-1} v_k &=& \dt^{k-1} f \qquad \mbox{ on } \pO
\\
\Delta^{k-2} v_k &=& 0 \qquad \mbox{ on } \pO
\\
\dots &&\dots
\\
v_k &=& 0 \qquad \mbox{ on } \pO~.
\eea
Even $p=3$ seems impractical in $d=3$ dimensions.

In a nonlinear coupled time-marching setting,
$\dot f$, $\ddot f$, etc, could be approximated via a low-order
BDF.

\subsection{Implementation III: eigenmodes all on the boundary (sketch)}

This is far-fetched, and only possible for spatially smooth enough $f$,
and a fixed shape.
It will have an expensive precomputation phase,
but perhaps fast evolution compared to the above,
especially for larger $\eps$ where a large order $p$ would help.
Find the boundary data of all Dirichlet eigenmodes $\dn \phi_j$
for eigenvalues $\lambda_j \le E$, for some convergence parameter $E$.
Weyl tells us there are $N=\bigO(E^{d/2})$ of them.

*** TO DO

The point is that the Laplace and polyharmonic BVP solves
are all equally
simple sums of rank-1 projectors in $L^2(\pO)$, in the eigenbasis.
No BIE quadratures, close evaluations, or biharmonic are needed.

*** study convergence rate, may be unacceptable.



% NNNNNNNNNNNNNNNNNNNNNNNNNNNNNNNNNNNNNNNNNNNNNNNNNNNNNNNNNNNNNNNNNNNNNNNNNNN
\section{The Neumann case}

For a given physical flux boundary data $f: \pO\times(0,T)\to\R$ entering the domain,
we have, by Fick's law at the boundary, the IBVP,
\bea
\eps\dot{u} - \Delta u &=& 0    \qquad \mbox{ in } \Omega\times (0,T)
\label{pden}
\\
\dn u  &=& \eps f  \qquad \mbox{ on } \pO \times (0,T)
\label{bcn}
\\
u(\cdot, 0)    &=&  0 \qquad \mbox{ in } \Omega~,
\label{icn}
\eea
where $\dn = \mbf{n}\cdot\nabla$ and $\mbf{n}$ is the unit surface
normal facing out of $\Omega$.
$f(x,t)$ is taken to be sufficiently smooth in $x$ and $t$.
It is standard that the IBVP has a unique solution.
Generally $\int_\pO f \neq 0$ so the total heat content $\int_\Omega u$
changes at a $\bigO(1)$ rate, independent of $\eps$.
Specifically, fixing $\eps>0$,
\be
\dt \int_\Omega u = \int_\pO f(\cdot,t)~, \qquad t\in(0,T)
\label{cons}
\ee
which easily follows from \eqref{pden} and the divergence theorem.
Nonzero initial conditions for \eqref{icn} would be easy to include,
since they can be replaced by their spatial average after $t>\bigO(\eps)$
because all fluctuations decay exponentially (see other singular perturbation
notes).

To analyze how close our asymptotic corrections are to the solution, we
need the following analogue of Lemma~\ref{l:gstab}, in which again $g$ is scaled to correspond to a physical volume heat source.

\begin{lem}[Stability to volume driving, zero mean Neumann case]
  Let $g: \Omega\times(0,T) \to\R$ be sufficiently smooth
  volume driving data obeying
  \be
  \int_\Omega g(\cdot,t)=0~, \qquad \forall t\in(0,T)
  \qquad\mbox{zero net heat (ZNH) condition~.}
  \label{znh}
  \ee
  Let $w$ solve the inhomogeneous IBVP
  \bea
  \eps\dot{w} - \Delta w &=& \eps g    \qquad \mbox{ in } \Omega\times (0,T)
  \\
  \dn w  &=& 0  \qquad \mbox{ on } \pO \times (0,T)
  \label{bcn0}
  \\
  w(\cdot, 0)    &=&  0 \qquad \mbox{ in } \Omega~.
  \eea
  Then, letting $\lambda_0^{(N)}<
  \lambda_1^{(N)} \le \lambda_2^{(N)} \le \dots$
  be the nondecreasing Neumann Laplace eigenvalues of $\Omega$,
  \be
  \|w(\cdot,t)\|_\LTO \;\le\;
  \frac{\eps}{\lambda_1^{(N)}} \sup_{t\in(0,T)} \|g(\cdot,t)\|_\LTO
  ~.
  \label{gstabn}
  \ee
  \label{l:gstabn}
\end{lem}
\begin{proof}
  Firstly, \eqref{znh} combined with the divergence theorem
  and the initial condition implies zero total heat,
  \be
  \int_\Omega w(\cdot,t) = 0 ~, \qquad t \in (0,T)~.
  \label{wzm}
  \ee
  The energy formula \eqref{energy} follows from the PDE
  as in the Dirichlet case.
  At each $t$, $w$ has the Neumann eigenfunction expansion
  $w(\cdot,t) = \sum_{j=1}^\infty a_j \phi_j^{(N)}$, where the
  trivial constant eigenfunction $\phi_0^{(N)}$ term
  is absent because its coefficient is zero by orthogonality and \eqref{wzm}.
  Since $w$ has compatible data \eqref{bcn0} and is smooth
  (since $g$ is smooth),
  both $w$ and $\Delta w$ have eigenfunction expansions convergent
  in $\LTO$, so
  $\int_\Omega w \Delta w = -\sum_{j=1}^\infty \lambda_j^{(N)} a_j^2 \le
  -\lambda_1^{(N)} L^2$,
  where, as before, $L(t) := \|w(\cdot,t)\|_\LTO$.
  The rest of the proof now follows as for Lemma~\ref{l:gstab}
  with $\lambda_1$ replaced by $\lambda_1^{(N)}$.
  \end{proof}

This bound of size $\bigO(\eps)$ is somewhat surprising, being 
much smaller than the input power of $\bigO(1)$;
it results because the heat rapidly diffuses
on an $\bigO(\eps)$ timescale.
Yet, in contrast to the Dirichlet case, no net heat is sucked away, hence
the need for ZNH.

{\bf Zeroth-order approximation.}
Returning to the IBVP \eqref{pden}--\eqref{icn},
define the net flux function of time,
\be
f_0(t) := \int_\pO f(\cdot,t)~, \qquad t\in(0,T)~.
\label{f0}
\ee
Define the spatially constant time-varying function
\be
v_0(x,t) := \frac{1}{\vol\Omega} \int_0^t f_0(s) ds
\qquad \forall x \in \Omega, \; t\in(0,T)~,
\ee
motivated by integrating \eqref{cons} from time 0 to $t$.
The zeroth-order solution is simply $u_0:=v_0$.
Its PDE residual is $(\eps \dt -\Delta)u_0 = \eps f_0(t) / \vol \Omega$,
yet (unlike the Dirichlet case), it also fails to satisfy the BCs \eqref{bcn}
any better than a trivial guess, since $\dn u_0 \equiv 0$.
This corresponds to the quasistatic approximation used by \cite{diegmiller18}.
Despite failing to match the BCs at all, its solution error is actually $\bigO(\eps)$.
We postpone that proof to below, since it turns out rather strangely to
piggyback off of the following 1st-order construction.

{\bf First-order correction ($p=2$).}
Now let $v_1$ solve, at each time $t$, the static Neumann Poisson BVP,
\bea
\Delta v_1 &=& \frac{1}{\vol\Omega} f_0 \qquad \mbox{ in } \Omega
\label{v1pde}
\\
\dn v_1 &=& f \qquad \mbox{ on } \pO~,
\\
\int_\Omega v_1 &=& 0~.
\label{v1zm}
\eea
This is consistent at each $t$ because the volume integral of the PDE
right-hand side equals the boundary integral
(the standard compatibility condition for a Neumann Poisson BVP).
The usual nullspace of constant functions is reduced to a unique solution
by the constraint \eqref{v1zm}.
Then the trial solution
$u_1 := v_0 + \eps v_1$
has, by cancellation, a residual $(\eps \dt -\Delta)u_1 = \eps^2 \dot v_1$,
and satisfies the BC \eqref{bcn} exactly.
Applying Lemma~\ref{l:gstabn} to $w=u_1 -u$, where $g=\eps \dot v_1$,
shows that $u_1$ is $\bigO(\eps^2)$ accurate.
For full rigor, this also needs the statement that $\dot v_1$ is
bounded; for H\"older norms this is provided by applying
a Schauder estimate \cite{nardi15} for the Neumann Poisson problem
with zero mean to \eqref{v1pde}--\eqref{v1zm}, or, specifically
its time-derivative.

{\bf Rigorous error bound for zeroth-order approximation.}
Applying the same Schauder estimate \cite{nardi15}
to\eqref{v1pde}--\eqref{v1zm}, shows that,
in H\"older norms,
$\|v_1\|_\Omega \le C_\Omega \|f\|_\pO$,
which is $\bigO(1)$ by assumption on the IBVP.
But, by the triangle inequality $\|u_0-u\| \le \|u_1-u_0\| + \|u_1-u\|
=\bigO(\eps) + \bigO(\eps^2) = \bigO(\eps)$
using that $u_1-u_0 = \eps v_1$ and the above error estimate
on $u_1$.
Thus finally we have shown that the zeroth-order approximation
is $\bigO(\eps)$ accurate, with constants depending on $\Omega$ and the
data $f$.


{\bf Higher-order correction.}
With $v_1$ solved,
one may continue to arbitrary order using homogeneous Neumann BCs
from now on.
For instance, $v_2$ would solve, at each time $t$,
\bea
\Delta v_2 &=& \dot v_1 \qquad \mbox{ in } \Omega
\\
\dn v_2 &=& 0 \qquad \mbox{ on } \pO~,
\\
\int_\Omega v_2 &=& 0~.
\eea
This is compatible, by taking the time-derivative of \eqref{v1zm}.
The approximation $u_2:=v_0+\eps v_1 + \eps^2 v_2$ is be $\bigO(\eps^3)$
accurate, by applying Lemma~\ref{l:gstabn} to $w = u_2-u$.
The numerical implementation would involve now a biharmonic solve,
as in the Dirichlet case one order lower.
The error analysis easily extends to any fixed asymptotic order.



\subsection{Implementation sketch}

$v_1$ may be solved using an analytic particular solution to \eqref{v1pde},
such as $(f_0(t)/d \vol\Omega)\,\|x\|^2$, leaving
only a static Laplace Neumann BVP to solve to correct the boundary data.
Volume integrals needed to impose \eqref{v1zm} can all be pushed
to the boundary using Green's identities.
Thus it would be rather easy to test the first-order correction scheme.

Thus only a {\em single} Laplace Neumann BVP per timestep is needed
to achieve $\bigO(\eps^2)$ accuracy; this is much simpler than
the Dirichlet case.

TO CONTINUE ***


\subsection{Example application to cell polarization nonlinear coupled IBVP}


TO DO ***

(Write an $\bigO(\delta t)$ fully implicit scheme combining
the surface solve and the interior Laplace solve,
using the notation of the outer solution in the other set of notes on
singular perturbation.)






\section{Discussion}

\bi
\item
  Clearly finite-difference schemes with explicit timestepping are
  a disaster, since $\delta t \le c \eps h^2$.
  Implicit is worth exploring; in fact higher-order
  implicit schemes may be related to the correction iteration \eqref{up}.
\item
  Taking arbitrarily high time derivatives of the BC data $f$ is dangerous
  and loses digits. A heat-potential scheme may not lose these digits,
  but is much more complicated to get correct.
\item
Strangely, if $f(x,\cdot) \in C^\infty((0,T))$,
taking $p\to\infty$ in \eqref{upformal} gives the formal Neumann series
\be
u_\infty := \Lap_D f + \eps \Delta_0^{-1} \dt \Lap_D f + \eps^2 \Delta_0^{-2} \dt^2 \Lap_D f + \dots
= (I - \eps \Delta_0^{-1} \dt)^{-1} \Lap_D f
\ee
whose meaning is obscure.
It is not a mere translation in time since it has no
reciprocal factorial factors. Its convergence a mystery
even for small $\eps$, because $\dt$ is not a bounded operator
unless we can restrict $f$ to some fancy very smooth function space.
\item
  I have tried analogous $\eps$-expansions for heat potentials
  to solve the Dirichlet IBVP.
  However, the mass of the algebraic tails in time decays like
  fractional powers of $\eps$, and it is not obvious how to
  write the result as a perturbation from the quasistatic solution $u_1$.
  I am far from a numerical method there. In contrast, the above seems
  more natural since it exploits the short $\bigO(\eps)$ memory of
  the domain problem.
\ei


% BBBBBBBBBBBBBBBBBBBBBBBBBBBBBBBBBBBBBBBBBBBBBBBBBBBBBBBBBBBBBBBBBBBBBBBBBBBB
\bibliographystyle{abbrv}
\bibliography{localrefs}
\end{document}

