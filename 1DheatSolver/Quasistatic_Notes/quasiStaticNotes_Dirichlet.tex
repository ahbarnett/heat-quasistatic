\documentclass[11pt]{article}

\usepackage{mathtools,amsmath,amssymb}
\usepackage{graphicx,float}
\usepackage{calc}
\usepackage[margin=1.in]{geometry}
\usepackage[usenames]{color}
\usepackage{fancyvrb}
\usepackage{hyperref}
\hypersetup{
	colorlinks=true, linkcolor=blue, filecolor=magenta, urlcolor=cyan, citecolor=gray, bookmarks=true, pdfpagemode=FullScreen,
}
\usepackage{tikz}
%\setlength\parindent{0pt}
\numberwithin{equation}{section}
\title{Implementation of a High-Diffusivity Heat Equation Scheme\vspace{-3ex}}
%\author{\vspace{-1.5in}}

\newcommand{\Lap}[1]{\text{Lap}_{D}#1}
\newcommand{\lap}[1]{\Delta_{0}^{-1}#1}
\newcommand{\pt}{\partial_t}
\newcommand{\dt}{\Delta t}
\newcommand{\beq}{\begin{equation}}
\newcommand{\eeq}{\end{equation}}
\newcommand{\dx}{\Delta x}
\newcommand{\note}[1]{{\color{red} #1}}


\begin{document}
\maketitle	

\section{Introduction}
In this document, we implement the scheme presented in \cite{quasistaticNotes} to second order accuracy using the 1D domain $\Omega = [0,1]$. We compare this implementation to one that uses Rothe's method and second-order accurate centered finite-differences in space. 

After presenting the problem, we give the order 2 accurate scheme using quasistatic corrections. We then discuss Rothe's method. We finally present numerical examples and a comparison. 

\section{The problem}
Let $u(x,t)$ solve the following heat equation problem on the domain $\Omega = [0,1]$

\begin{equation}\label{eqn:IBVP}
\begin{cases}
\epsilon \dot{u} - \Delta u = 0, & x \in\Omega\times(0,T], \\ 
u = f, & x \in\partial \Omega\times (0,T], \\
u(x,0) = 0, & x\in\Omega.
\end{cases}
\end{equation}
Here, $\epsilon\ll 1$, $T$ is a final time, and $f$ represents known Dirichlet data.
\section{The high-order accurate scheme using quasistatic corrections}
The notes in \cite{quasistaticNotes} develop an $\mathcal{O}(\epsilon^p)$ accurate solution to the IBVP in \eqref{eqn:IBVP} given by 

\begin{equation}
u_p = \sum_{k = 0}^{p-1} \epsilon^k \Delta_0^{-k}\Lap\partial_t^{k}f, 
\end{equation}
where the operators $\Lap f$ and $\lap g$ solve the problems 

\begin{equation*}\begin{cases}
\Delta u = 0, & x \in\Omega\times(0,T], \\ 
u = f, & x \in\partial \Omega\times (0,T], \\
u(x,0) = 0, &x\in\Omega,
\end{cases} \qquad \text{and} \qquad 
  \begin{cases}
\Delta u = g, & x \in\Omega\times(0,T], \\
u = 0, & x \in\partial \Omega\times (0,T], \\
u(x,0) = 0, &x\in\Omega,
\end{cases}
\end{equation*}
respectively.
\section{The second-order accurate solution}\label{sec:O2sol}
Consider the order-two-accurate solution to the IBVP in \eqref{eqn:IBVP} using quasi-static corrections
\begin{equation}\label{eqn:quasistaticScheme}
u_2 = v_0 + \epsilon \Delta_0^{-1}\dot v_0, 
\end{equation}
where $v_0\equiv \Lap f$ and $\dot v_0\equiv\Lap\partial_t f$.

To implement \eqref{eqn:quasistaticScheme}, we use the algorithm proposed in \cite{quasistaticNotes} as follows 
\begin{enumerate}
\item For each time step $j$, fill $w_0(\cdot, t_j) = \Lap {f(\cdot, t_j)}$ and $w_1(\cdot, t_j) = \lap{w_0(\cdot, t_j)}$. 
\item Compute $v_0(\cdot, t_j) = w_0(\cdot, t_j)$ and $v_1(\cdot, t_j) = \pt w_1(\cdot, t_j)$ using an order 2 accurate finite-difference formula. 
\item Compute $u_2 = v_0 + \epsilon v_1$ at each node and each time-step. 
\end{enumerate}

In the algorithm above, we approximate $\pt w_1 \approx (w_1(t + \dt) - w_1(t - \dt))/2\dt$; we need to choose an appropriate value for the time step $\dt$ in the finite-difference formula. For an implementation at a final time $T$, we need two elliptic solves at each of $t = T$, $T - \dt$ and $T + \dt$, a total of 6 elliptic solves.  

%If we are looking for a solution at the final time $T$, we try an alternate implementation that does not require approximating time derivatives using FD formulas:
%\begin{enumerate}
%\item Find $v_0$ by solving $\Lap f(\cdot, T)$. 
%\item Find $\dot v_0$ by solving $\Lap \pt f(\cdot, T)$. 
%\item Find $v_1$ by solving $\lap{\dot{v}_0}$. 
%\end{enumerate}
%The total number of elliptic solves in the above computation is 3. 

\section{Rothe's method}
Rothe's method applies a temporal discretization to the differential equation in \eqref{eqn:IBVP} to arrive at a sequence of elliptic equations which are then solved using boundary integral methods.\\ 

Leaving the space variable continuous, we discretize the heat equation in \eqref{eqn:IBVP} in time using BDF2. Let $t^n = n\dt$, with $n = 0, \dots, N_t$, $u^n \approx u(\cdot, t^n)$. Here, $N_t$ represents the number of time steps. We have
\begin{equation*}
u^{n+1} - \frac{4}{3}u^{n} + \frac{1}{3}u^{n - 1}  = \frac{2}{3}\frac{\dt}{\epsilon}(u_{xx})^{n+1}, 
\end{equation*}
so that 
\begin{equation}\label{eqn:ellipticEqns}
(u_{xx})^{n+1} - \frac{3\epsilon}{2\dt}u^{n+1} = \frac{-2\epsilon}{\dt}u^{n} + \frac{\epsilon}{2\dt}u^{n - 1}.
\end{equation}
At each time-step, we solve an elliptic equation to find $u^{n+1}$, a total of $N_t$ elliptic solves.\\

We approximate the solution at the first time-step $t = \dt$ using 
\beq
u(x, \dt) = u_0(x) +\frac{\dt}{\epsilon\dx^2}(u_0(x + \dx) - 2u_0(x) + u_0(x - \dx) ) + \mathcal{O}(\dt^2), \qquad \text{as }\dt\rightarrow 0. 
\eeq
\section{Implementation}\label{sec:implementation}
We carry out numerical implementations of Rothe's Method and the Quasistatic asymptotic method using three exact solutions of \eqref{eqn:IBVP} as test cases.
\subsection{Example 1}
For this implementation, we use the exact solution  
\begin{equation}
u_{e}(x,t) = A \cos(\sqrt{\epsilon} x + a)e^{-t}, \qquad x\in[0,1], t\in[0,T], 
\end{equation}
for testing. 
We set $A = 200$, $a = -1.123$ and $T = 1$. We use second-order centered finite-differences to solve the elliptic equations in \eqref{eqn:ellipticEqns} rather than boundary integral methods (at least for this preliminary test), and perform a grid refinement study. We measure the error using the maximum norm. \\

For $\epsilon = 10^{-4}$ We obtain the following results using Rothe's method
\begin{footnotesize}
\begin{Verbatim}[frame = single]
Using Rothe method for epsilon = 1.0e-04
t=1.0000e+00: Nx= 10 Nt=  10 dt=1.000e-01 maxErr=1.45e-06
t=1.0000e+00: Nx= 20 Nt=  20 dt=5.000e-02 maxErr=3.48e-07 order=2.05e+00
t=1.0000e+00: Nx= 40 Nt=  40 dt=2.500e-02 maxErr=8.54e-08 order=2.03e+00
t=1.0000e+00: Nx= 80 Nt=  80 dt=1.250e-02 maxErr=2.12e-08 order=2.01e+00
t=1.0000e+00: Nx=160 Nt= 160 dt=6.250e-03 maxErr=5.27e-09 order=2.01e+00
\end{Verbatim}
\end{footnotesize}
In the output table above, $\texttt{Nx}$ refers to the number of grid points, $\texttt{Nt}$ denotes the number of time-steps,  $\texttt{dt}$ corresponds to the time step $\dt$, $\texttt{maxErr}$ is the error in the maximum norm and $\texttt{order}$ denotes the overall order of accuracy. The same follows for the other tables presented in this section.  

For the Quasistatic asymptotic method, we obatin
\begin{footnotesize}
\begin{Verbatim}[frame = single]
Using Quasistatic method for epsilon = 1.0e-04
t=1.0000e+00: Nx= 10 Nt=  10 dt=1.000e-01 maxErr=6.67e-07
t=1.0000e+00: Nx= 20 Nt=  20 dt=5.000e-02 maxErr=1.63e-07 order=2.03e+00
t=1.0000e+00: Nx= 40 Nt=  40 dt=2.500e-02 maxErr=3.77e-08 order=2.12e+00
t=1.0000e+00: Nx= 80 Nt=  80 dt=1.250e-02 maxErr=6.29e-09 order=2.59e+00
t=1.0000e+00: Nx=160 Nt= 160 dt=6.250e-03 maxErr=1.58e-09 order=1.99e+00
\end{Verbatim}
\end{footnotesize}

For $\epsilon = 10^{-6}$, and using Rothe's method, we find

\begin{footnotesize}
\begin{Verbatim}[frame = single]
Using Rothe method for epsilon = 1.0e-06
t=1.0000e+00: Nx= 10 Nt=  10 dt=1.000e-01 maxErr=1.43e-08
t=1.0000e+00: Nx= 20 Nt=  20 dt=5.000e-02 maxErr=3.45e-09 order=2.05e+00
t=1.0000e+00: Nx= 40 Nt=  40 dt=2.500e-02 maxErr=8.46e-10 order=2.03e+00
t=1.0000e+00: Nx= 80 Nt=  80 dt=1.250e-02 maxErr=2.16e-10 order=1.97e+00
t=1.0000e+00: Nx=160 Nt= 160 dt=6.250e-03 maxErr=6.72e-11 order=1.69e+00
\end{Verbatim}
\end{footnotesize}
while the quasistatic approach yields the following results
\begin{footnotesize}
\begin{Verbatim}[frame = single]
Using Quasistatic method for epsilon = 1.0e-06
t=1.0000e+00: Nx= 10 Nt=  10 dt=1.000e-01 maxErr=6.65e-09
t=1.0000e+00: Nx= 20 Nt=  20 dt=5.000e-02 maxErr=1.66e-09 order=2.00e+00
t=1.0000e+00: Nx= 40 Nt=  40 dt=2.500e-02 maxErr=4.14e-10 order=2.00e+00
t=1.0000e+00: Nx= 80 Nt=  80 dt=1.250e-02 maxErr=1.03e-10 order=2.01e+00
t=1.0000e+00: Nx=160 Nt= 160 dt=6.250e-03 maxErr=2.03e-11 order=2.34e+00
\end{Verbatim}
\end{footnotesize}
Fixing the number of grid point to 100 and setting $\dt = 0.1$, we also generate the following results for Rothe's method
\begin{footnotesize}
\begin{Verbatim}[frame = single]
Using Rothe method
eps = 1.0000e-02, t=1.0000e+00, Nx=100 Nt=  10 dt=1.000e-01 maxErr=1.58e-04
eps = 1.0000e-03, t=1.0000e+00, Nx=100 Nt=  10 dt=1.000e-01 maxErr=1.48e-05
eps = 1.0000e-04, t=1.0000e+00, Nx=100 Nt=  10 dt=1.000e-01 maxErr=1.45e-06
eps = 1.0000e-05, t=1.0000e+00, Nx=100 Nt=  10 dt=1.000e-01 maxErr=1.44e-07
eps = 1.0000e-06, t=1.0000e+00, Nx=100 Nt=  10 dt=1.000e-01 maxErr=1.43e-08
eps = 1.0000e-07, t=1.0000e+00, Nx=100 Nt=  10 dt=1.000e-01 maxErr=1.44e-09
eps = 1.0000e-08, t=1.0000e+00, Nx=100 Nt=  10 dt=1.000e-01 maxErr=1.53e-10
eps = 1.0000e-09, t=1.0000e+00, Nx=100 Nt=  10 dt=1.000e-01 maxErr=1.16e-11
eps = 1.0000e-10, t=1.0000e+00, Nx=100 Nt=  10 dt=1.000e-01 maxErr=1.91e-12
eps = 1.0000e-11, t=1.0000e+00, Nx=100 Nt=  10 dt=1.000e-01 maxErr=1.10e-11
\end{Verbatim}
\end{footnotesize}
It appears that Rothe's method remains accurate as $\epsilon$ becomes smaller and $\dt\gg\epsilon$. 

\subsection{Example 2}
For this example, we use the exact solution 
\beq
u_e(x,t) = ax^2 + bx + c + \frac{2a}{\epsilon} t, \qquad x \in [0,1], t\in[0,T], 
\eeq
with  $a = \epsilon/2$, $b = 23$ and $C = 123$. 
For $\epsilon = 10^{-4}$ we obtain
\begin{footnotesize}
\begin{Verbatim}[frame = single]
Using Rothe method for epsilon = 1.0e-04
t=1.0000e+00: Nx= 10 Nt=  10 dt=1.000e-01 maxErr=1.71e-13
t=1.0000e+00: Nx= 20 Nt=  20 dt=5.000e-02 maxErr=2.74e-12 order=-4.01e+00
t=1.0000e+00: Nx= 40 Nt=  40 dt=2.500e-02 maxErr=6.17e-12 order=-1.17e+00
t=1.0000e+00: Nx= 80 Nt=  80 dt=1.250e-02 maxErr=2.73e-11 order=-2.14e+00
t=1.0000e+00: Nx=160 Nt= 160 dt=6.250e-03 maxErr=1.15e-10 order=-2.08e+00
\end{Verbatim}
\end{footnotesize}
\begin{footnotesize}
\begin{Verbatim}[frame = single]
Using Quasistatic method for epsilon = 1.0e-04
t=1.0000e+00: Nx= 10 Nt=  10 dt=1.000e-01 maxErr=4.97e-13
t=1.0000e+00: Nx= 20 Nt=  20 dt=5.000e-02 maxErr=1.96e-12 order=-1.98e+00
t=1.0000e+00: Nx= 40 Nt=  40 dt=2.500e-02 maxErr=4.48e-12 order=-1.19e+00
t=1.0000e+00: Nx= 80 Nt=  80 dt=1.250e-02 maxErr=9.66e-12 order=-1.11e+00
t=1.0000e+00: Nx=160 Nt= 160 dt=6.250e-03 maxErr=4.60e-11 order=-2.25e+00
\end{Verbatim}
\end{footnotesize}
This solution is a linear combination of a degree 2 polynomial in $x$ and degree 1 polynomial in $t$. We expect the numerical solution to be accurate to machine precision times the condition number of the underlying matrix system being inverted. 

\subsection{Example 3}
\note{Work in progress...}\\
In this example, we plan to find a test solution of \eqref{eqn:IBVP} for which the boundary conditions are oscillatory functions. That is, we find a solution to the problem 
\beq\label{eq:heatBCs}
\begin{cases}
\epsilon u_{t} - u_{xx} = 0, & x\in(0,1),\\
u(0,t) = A(t),\\
u(1,t) = B(t), \\
u(x, 0) = u_0(x), & x\in [0,1], 
\end{cases}
\eeq
where $A(t)$ and $B(t)$ may be of the form $\alpha \cos(\beta t)$, for example. \\

We solve \eqref{eq:heatBCs} by setting $u(x,t) = U(x,t) + K(x,t)$, so that $U$ solves the following problem 
\beq\label{eq:heatNoBC}
\begin{cases}
\epsilon U_{t} - U_{xx} = \underbrace{-\epsilon K_t + K_{xx}}_{q(x,t)} , & x\in(0,1),\\
U(0,t) = 0,\\
U(1,t) = 0, \\
U(x, 0) = \underbrace{u_0(x) - K(x,0)}_{U_{0}(x)}, & x\in [0,1]. 
\end{cases}
\eeq
The simplest function $K$ for which the \eqref{eq:heatNoBC} holds takes the form 
\beq
K(x,t) = (B(t) - A(t)) x + A(t). 
\eeq
We seek a solution of $U(x,t)$ to \eqref{eq:heatNoBC} as the series 
\beq
U(x,t) = \sum_{n = 1}^{\infty} T_{n}(t)F_{n}(x), 
\eeq
where 
\beq
F_n(x) = \sin\omega_n x, \quad \lambda_n = \omega_n^2 = n^2\pi^2, 
\eeq
and for $D = 1/\epsilon$, 
\beq
T_n(t) = T_n(0)e^{-\lambda_n D t} + \int_{0}^t\hat{q}_n(\tau)e^{-\lambda_nD(t - \tau)}d\tau. 
\eeq
Here, 
\beq
T_n(0) = 2\int_{0}^1U_{0}(x)F_n(x)dx, 
\eeq
and 
\beq
\hat q_n(t) = 2\int_{0}^1q(x,t)F_n(x). 
\eeq
\section{Conclusion and remarks}
Using examples of exact solutions to \eqref{eqn:IBVP}, we implemented the quasistatic asymptotic scheme developed in \cite{quasistaticNotes} and compared it to an implementation using Rothe’s method. Based on the results in section \ref{sec:implementation}, we write down the following conclusions: 
\begin{itemize}
\item The quasistatic asymptotic method appears to yield smaller errors than Rothe's method. 
\item Rothe's method remains accurate for $\dt\gg \epsilon$ as $\epsilon \rightarrow 0$. 
\end{itemize}
	 \clearpage
\bibliographystyle{plain}
\bibliography{refs}

  \end{document}
  
 %\begin{footnotesize}
%\begin{Verbatim}[frame = single]
%
%\end{Verbatim}
%\end{footnotesize}

%	\begin{figure}[H]
%	\centering
%	\includegraphics[width=8cm]{Option3Computed}
%	\includegraphics[width=8cm]{Option3Error}
%	\caption{Option 3 Implementation}
%\end{figure}

%\begin{algorithm}[H]
%	\caption{algorithm caption here }
%	\For{$ \nu = 1, 2 $ }{
%	}
%\end{algorithm}

% =====================================================
% =====================================================
% =====================================================









