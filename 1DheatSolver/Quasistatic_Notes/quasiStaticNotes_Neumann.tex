\documentclass[11pt]{article}

\usepackage{mathtools,amsmath,amssymb}
\usepackage{graphicx,float}
\usepackage{fancyvrb}
\usepackage{calc}
\usepackage[margin=1.in]{geometry}
\usepackage[usenames]{color}
\usepackage{hyperref}
\hypersetup{
	colorlinks=true, linkcolor=blue, filecolor=magenta, urlcolor=cyan, citecolor=gray, bookmarks=true, pdfpagemode=FullScreen,
}
\usepackage{tikz}
%\setlength\parindent{0pt}
\numberwithin{equation}{section}
\title{Rothe's Method with Neumann Conditions\vspace{-3ex}}
%\author{\vspace{-1.5in}}

\newcommand{\Lap}[1]{\text{Lap}_{D}#1}
\newcommand{\lap}[1]{\Delta_{0}^{-1}#1}
\newcommand{\pt}{\partial_t}
\newcommand{\pn}{\partial_n}
\newcommand{\dt}{\Delta t}
\newcommand{\dx}{\Delta x}
\newcommand{\dzx}{D_{0x}}
\newcommand{\Nx}{N_x}

\begin{document}
\maketitle	

\section{Introduction}
In this document, we implement Rothe's method to solve the 1D heat equation on the unit interval, with large diffusion coefficient, $1/\epsilon$, $\epsilon\ll 1$,  and non-zero Neumann BCs. We use order 2 accurate centered finite-differences to discretize in space and BDF2 formula to discretize in time. \\

We present a numerical example where the time-step is $\dt\gg\epsilon$ and perform a grid refinement study to check whether Rothe's method remains order 2 accurate as $\epsilon \rightarrow 0$. 
\section{The problem}
Let $u(x,t)$ solve the following heat equation problem on the domain $\Omega = [0,1]$, 

\begin{equation}\label{eqn:IBVP}
\begin{cases}
\epsilon \dot{u} - \Delta u = 0, & x \in\Omega\times(0,T], \\ 
\pn u = f, & x \in\partial \Omega\times (0,T], \\
u(x,0) = 0, & x\in\Omega.
\end{cases}
\end{equation}
Here, $\epsilon\ll 1$, $T$ is a final time, and $f$ represents known Neumann data. The notation $\pn u$ represents the derivatives in the direction of the normal vector at the boundary. 

\section{Rothe's method}
Rothe's method applies a temporal discretization to the differential equation in \eqref{eqn:IBVP} to arrive at a sequence of elliptic equations which are then solved using boundary integral methods.\\ 

Leaving the space variable continuous, we discretize the heat equation in \eqref{eqn:IBVP} in time using BDF2. Let $t^n = n\dt$, with $n = 0, \dots, N_t$, $u^n \approx u(\cdot, t^n)$. Here, $N_t$ represents the number of time steps. We have
\begin{equation*}
u^{n+1} - \frac{4}{3}u^{n} + \frac{1}{3}u^{n - 1}  = \frac{2}{3}\frac{\dt}{\epsilon}(u_{xx})^{n+1}, 
\end{equation*}
so that 
\begin{equation}\label{eqn:ellipticEqns}
(u_{xx})^{n+1} - \frac{3\epsilon}{2\dt}u^{n+1} = \frac{-2\epsilon}{\dt}u^{n} + \frac{\epsilon}{2\dt}u^{n - 1}.
\end{equation}
At each time-step, we solve an elliptic equation to find $u^{n+1}$, a total of $N_t$ elliptic solves. \\

We approximate the solution at the first time-step $t = \dt$ using 
\begin{equation}
u(x,\dt) = u_0(x) + \frac{\dt}{\epsilon\dx^2}(u_0(x + \dx) - 2u_0(x) + u_0(x - \dx)) + \mathcal{O}(\dt^2), \quad \text{as } \dt\rightarrow 0,
\end{equation}
where $\dx$ is the spatial step size. 

\section{Treatment of Neumann BCs}
Let $x_i = i\dx$ represent the discretization in space with $i = -1,0, \dots, N_x, N_x + 1$. Here we introduce one \emph{ghost} grid point at each end of the interval to treat the Neumann conditions using a centered discretization. Let $u_i^n~\approx~u(x_i,t^n)$. The discretization of the Neumann conditions at $x = 0$ and $x = 1$, takes the form 
\begin{align*}
	-\dzx u_{0}^n &= f_{l}(x, t^n), \quad x = 0,\\
		\dzx u_{\Nx}^n &= f_{r}(x, t^n),\quad x= 1, 
\end{align*}
where $\dzx u_i^n = (u_{i+1}^n - u_{i-1}^n)/(2\dx)$, $f_l$ and $f_r$ represent given Neumann data on the left and right boundaries respectively. We can then evaluate the solution at the ghost grid points using the following 
\begin{align*}
	u_{-1}^n &= u_1^n + 2\dx f_l(x,t^n), \quad x = 0, \\
	u_{\Nx + 1}^n &= u_{\Nx-1}^n + 2\dx f_r(x,t^n), \quad x = 1. 
\end{align*}
\section{Implementation}	
For the implementation, we use the exact solution  
\begin{equation}
u_{e}(x,t) = A \cos(\sqrt{\epsilon}x + a)e^{-t} + B \sin(\sqrt{\epsilon}x+a)e^{-t}, \qquad x\in[0,1], t\in[0,T], 
\end{equation}
for testing. 
We set $A = 200$, $B = 100$, $a = -1.123$, and $T = 1$. We use second-order centered finite-differences to solve the elliptic equations in \eqref{eqn:ellipticEqns}, and perform a grid refinement study. We measure the error using the maximum norm  

For $\epsilon = 10^{-4}$, we obtain the following results
\begin{footnotesize}
	\begin{Verbatim}[frame = single]
Using the Rothe method with Neumann BC and eps = 1.0e-04
t=1.0000e+00: Nx= 10 Nt=  10 dt=1.000e-01 maxErr=2.19e-02
t=1.0000e+00: Nx= 20 Nt=  20 dt=5.000e-02 maxErr=5.65e-03 order=1.95e+00
t=1.0000e+00: Nx= 40 Nt=  40 dt=2.500e-02 maxErr=1.43e-03 order=1.98e+00
t=1.0000e+00: Nx= 80 Nt=  80 dt=1.250e-02 maxErr=3.61e-04 order=1.99e+00
t=1.0000e+00: Nx=160 Nt= 160 dt=6.250e-03 maxErr=9.07e-05 order=1.99e+00
t=1.0000e+00: Nx=320 Nt= 320 dt=3.125e-03 maxErr=2.30e-05 order=1.98e+00
t=1.0000e+00: Nx=640 Nt= 640 dt=1.563e-03 maxErr=6.35e-06 order=1.85e+00
t=1.0000e+00: Nx=1280 Nt=1280 dt=7.813e-04 maxErr=1.05e-07 order=5.91e+00
	\end{Verbatim}
\end{footnotesize}
In the output table above, \texttt{Nx} refers to the number of grid points, \texttt{Nt} denotes the number of time-steps, \texttt{dt} corresponds to the time step $\dt$, \texttt{maxErr} is the error in the maximum norm and \texttt{order}
denotes the overall order of accuracy. The same follows for the other tables presented in this section.\\

The results confirm the expected order 2 accuracy of the scheme. \\

For $\epsilon = 10^{-6}$, we obtain the following results
\begin{footnotesize}
	\begin{Verbatim}[frame = single]
Using the Rothe method with Neumann BC and eps = 1.0e-06
eps=1.0e-06, t=1.0e+00: Nx=  10 Nt=  10 dt=1.0e-01 maxErr=3.10e-02, condNum=3.5e+08
eps=1.0e-06, t=1.0e+00: Nx=  20 Nt=  20 dt=5.0e-02 maxErr=7.99e-03, condNum=1.2e+09 order=1.95e+00
eps=1.0e-06, t=1.0e+00: Nx=  40 Nt=  40 dt=2.5e-02 maxErr=2.03e-03, condNum=4.6e+09 order=1.98e+00
eps=1.0e-06, t=1.0e+00: Nx=  80 Nt=  80 dt=1.3e-02 maxErr=5.14e-04, condNum=1.8e+10 order=1.98e+00
eps=1.0e-06, t=1.0e+00: Nx= 160 Nt= 160 dt=6.3e-03 maxErr=1.35e-04, condNum=7.0e+10 order=1.93e+00
eps=1.0e-06, t=1.0e+00: Nx= 320 Nt= 320 dt=3.1e-03 maxErr=1.92e-05, condNum=2.8e+11 order=2.81e+00
eps=1.0e-06, t=1.0e+00: Nx= 640 Nt= 640 dt=1.6e-03 maxErr=1.49e-05, condNum=1.1e+12 order=3.69e-01
eps=1.0e-06, t=1.0e+00: Nx=1280 Nt=1280 dt=7.8e-04 maxErr=6.13e-04, condNum=4.4e+12 order=-5.37e+00
eps=1.0e-06, t=1.0e+00: Nx=2560 Nt=2560 dt=3.9e-04 maxErr=3.24e-04, condNum=1.7e+13 order=9.22e-01
	\end{Verbatim}
\end{footnotesize}
In the above results, \texttt{condNum} refers to the condition number of the implicit matrix used to solve the elliptic equations numerically. \\

We see order 2 accurate convergence at coarse resolutions. The condition number increases as we increase the resolution, and thus the error increases at the finer resolutions. \\

For $\epsilon = 10^{-8}$, we obtain the following results
\begin{footnotesize}
	\begin{Verbatim}[frame = single]
Using the Rothe method with Neumann BC and eps = 1.0e-08
eps=1.0e-08, t=1.0e+00: Nx=  10 Nt=  10 dt=1.0e-01 maxErr=3.19e-02, condNum=3.5e+10
eps=1.0e-08, t=1.0e+00: Nx=  20 Nt=  20 dt=5.0e-02 maxErr=8.24e-03, condNum=1.2e+11 order=1.95e+00
eps=1.0e-08, t=1.0e+00: Nx=  40 Nt=  40 dt=2.5e-02 maxErr=2.11e-03, condNum=4.6e+11 order=1.96e+00
eps=1.0e-08, t=1.0e+00: Nx=  80 Nt=  80 dt=1.3e-02 maxErr=5.82e-04, condNum=1.8e+12 order=1.86e+00
eps=1.0e-08, t=1.0e+00: Nx= 160 Nt= 160 dt=6.3e-03 maxErr=2.70e-04, condNum=7.0e+12 order=1.11e+00
eps=1.0e-08, t=1.0e+00: Nx= 320 Nt= 320 dt=3.1e-03 maxErr=3.24e-03, condNum=2.8e+13 order=-3.58e+00
eps=1.0e-08, t=1.0e+00: Nx= 640 Nt= 640 dt=1.6e-03 maxErr=3.89e-03, condNum=1.1e+14 order=-2.63e-01
eps=1.0e-08, t=1.0e+00: Nx=1280 Nt=1280 dt=7.8e-04 maxErr=4.92e-02, condNum=4.4e+14 order=-3.66e+00
eps=1.0e-08, t=1.0e+00: Nx=2560 Nt=2560 dt=3.9e-04 maxErr=1.66e-01, condNum=1.7e+15 order=-1.75e+00
	\end{Verbatim}
\end{footnotesize}
Decreasing the value of $\epsilon$ appears to cause the condition number of the matrix used in the solve to become worse at the finer resolutions. \\

We fix $\Nx = 100$, $\dt=0.1$, vary $\epsilon$ and measure the error 
\begin{footnotesize}
	\begin{Verbatim}[frame = single]
Using the Rothe method with Neumann BC
eps=1.0e-02, t=1.0000e+00: Nx=100 Nt=  10 dt=1.000e-01 maxErr=6.89e-02
eps=1.0e-03, t=1.0000e+00: Nx=100 Nt=  10 dt=1.000e-01 maxErr=8.08e-05
eps=1.0e-04, t=1.0000e+00: Nx=100 Nt=  10 dt=1.000e-01 maxErr=2.19e-02
eps=1.0e-05, t=1.0000e+00: Nx=100 Nt=  10 dt=1.000e-01 maxErr=2.88e-02
eps=1.0e-06, t=1.0000e+00: Nx=100 Nt=  10 dt=1.000e-01 maxErr=3.10e-02
eps=1.0e-07, t=1.0000e+00: Nx=100 Nt=  10 dt=1.000e-01 maxErr=3.17e-02
eps=1.0e-08, t=1.0000e+00: Nx=100 Nt=  10 dt=1.000e-01 maxErr=3.21e-02
eps=1.0e-09, t=1.0000e+00: Nx=100 Nt=  10 dt=1.000e-01 maxErr=3.19e-02
eps=1.0e-10, t=1.0000e+00: Nx=100 Nt=  10 dt=1.000e-01 maxErr=1.61e-02
eps=1.0e-11, t=1.0000e+00: Nx=100 Nt=  10 dt=1.000e-01 maxErr=4.73e-02
eps=1.0e-12, t=1.0000e+00: Nx=100 Nt=  10 dt=1.000e-01 maxErr=6.26e-01
	\end{Verbatim}
\end{footnotesize}
Unlike the case with Dirichlet boundary conditions where the error decreases as $\epsilon$ decreases, the case with a Neumann condition appears to settle at an error of order $10^{-2}$ as we decrease the value of $\epsilon$. 

\section{Conclusion and remarks}
Using a simple 1D test case, we test the effectiveness of Rothe's method when the diffusion coefficient becomes very large, i.e. $1/\epsilon$, $\epsilon\ll 1$. We find that the performance of the Rothe's method becomes worse as $\epsilon$ decreases and this may be due to the worsening of the condition number of the implicit matrix used in the solve. \\

\subsection{Next steps: }
\begin{itemize}
\item Use boundary integral equations methods to solve the resulting equations after the time discretization in Rothe's method \eqref{eqn:ellipticEqns} instead of finite differences. Would that improve the convergence as $\epsilon$ becomes small? 
\item Implement the Quasi-static asymptotic method with a Neumann boundary condition and see if there is an improvement in the error for small epsilon. 
\end{itemize}
	 \clearpage
\bibliographystyle{plain}
\bibliography{refs}

  \end{document}
%	\begin{figure}[H]
%	\centering
%	\includegraphics[width=8cm]{Option3Computed}
%	\includegraphics[width=8cm]{Option3Error}
%	\caption{Option 3 Implementation}
%\end{figure}

%\begin{algorithm}[H]
%	\caption{algorithm caption here }
%	\For{$ \nu = 1, 2 $ }{
%	}
%\end{algorithm}

% =====================================================
% =====================================================
% =====================================================









